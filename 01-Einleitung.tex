\section{Einleitung}
\label{sec:Einleitung}

\subsection{Einführung in das Thema und seine Bedeutung}
Durch Fälle wie Edward Snowden und Mark Felt hat das Thema „Whistleblowing“ in den letzten Jahren zunehmend an Relevanz gewonnen.
Durch ersteren kamen den die Überwachungs- und Spionagepraktiken des US-Geheimdienstes NSA ans Licht.
Letzterer stellte eine Schlüsselfigur in der Watergate-Affäre dar, durch die die Vertuschung und versuchte Behinderung der Ermittlungen durch hohe Amtsträger, nachdem das Watergate Hotel eingebrochen war, publik wurde.
Zudem erlangt das Thema durch die Enthüllungsplattform „WikiLeaks“ weitere Relevanz.
Diese wurde 2006 gegründet und auf ihr wurden seit dem zahlreiche geheime Dokumente anonym veröffentlicht.\\
Obwohl die bekanntesten Whistleblower Fälle aus den USA kommen, ist das Thema nicht nur dort bedeutsam.
Auch in der Europäischen Union kommt es vermehrt zu Fällen von Whistleblowing.
Ein Beispiel aus den letzten Jahren wären die „Luxemburg-Leaks“.
Dort wurden 28.000 Seiten an Dokumenten der Luxemburger Steuerbehörde veröffentlicht, in denen mehreren Konzernen niedrige Steuern zugesichert wurden.
Hier wird ein durchaus positiver Effekt des Whistleblowing deutlich, nämlich dass erhebliche Missstände in diversen Unternehmen und Regierungen aufgedeckt werden können.
Die Whistleblower setzen sich mit Bekanntmachung der Unterlagen jedoch regelmäßig der Gefahr von Repressalien aus.
Hierunter fällt nicht nur eine mögliche Kündigung, sondern in vielen Fällen auch Strafverfolgung wobei teils erhebliche Strafen drohen.\\
Abhilfe hierfür und Schutz für die Hinweisgeber soll durch die Richtlinie 2019/1937 der EU gewährleistet werden.
Auf diese wird im Folgenden genauer eingegangen.

\subsection{Gliederung und Zielsetzung der Arbeit}
Diese Arbeit beschäftigt sich zunächst mit den wesentlichen Aussagen der EU-Whistleblower Richtlinie. 
Im Rahmen dessen wird darauf eingegangen, was unter Whistleblowing überhaupt zu verstehen ist und wie die Entwicklungen in den USA die Verabschiedung der Richtlinie mit beeinflusst haben.
Im Anschluss werden die verschiedenen Herausforderungen beleuchtet, die mit dem Whistleblowing einhergehen.
Vor allem wird sich hier mit dem arbeits-, daten- und strafrechtlichen Problemen beschäftigt.
Zuletzt wird erörtert, ob durch das Whistleblowing Straftaten innerhalb von Unternehmen verhindert werden können.\\
Mit dieser Arbeit sollen sowohl die Herausforderungen als auch die Chancen beleuchtet werden, die sich durch das Whistleblowing ergeben. 
Zudem soll geklärt werden, ob die neue Richtlinie bei der wachsenden Problematik des Whistleblowings Abhilfe verschaffen kann.
