\section{Fazit}
\label{sec:Fazit}
Die Richtlinie (EU) 019/1937 des Europäischen Parlaments und des Rates zum Schutz von Personen, die Verstöße gegen das Unionsrecht melden, leistet einen wichtigen Beitrag Verstöße gegen geltendes Recht aufzuklären.
Durch sie werden die sogenannten Whistleblower geschützt, indem Unternehmen ein Verbot von jeglichen Repressalien gegenüber Hinweisgebern auferlegt wird.
Verstärkt wird dieser Schutz dadurch, dass das Unternehmen und nicht der Arbeitnehmer nachweisen muss, dass eine arbeitsrechtliche Maßnahme nicht infolge einer getätigten Meldung erfolgt ist.
Der Schutzbereich der Richtlinie ist ebenfalls sehr weit.
Geschützt wird jede Person, die irgendwie im Zusammenhang mit dem Unternehmen steht, jedoch muss diese Person bei Abgabe des Hinweises im guten Glauben gewesen sein, dass ihr Hinweis zutrifft und das Gemeldete einen Verstoß gegen geltendes Recht darstellt.\\
Mit der Implementierung gingen jedoch auch einige Herausforderungen einher.
Für Arbeitgeber stellt die Pflicht zur Errichtung von internen Meldekanälen ein erheblicher Aufwand und Kosten dar.
Zudem treffen ihn vermehrt Dokumentationspflichten, um gegebenenfalls nachweisen zu können, dass eine Maßnahme unabhängig von einem vorher gegebenen Hinweise getroffen wurde.
Hiermit einhergehen Herausforderungen bezüglich der Datenschutzgrundverordnung.
Personenbezogene Daten dürfen nur für einen bestimmten Zeitraum gespeichert werden.
Dies kann durch eine Anonymisierung der Akte umgangen werden.
Auch stellt das Offenlegen von Geschäftsgeheimnissen eine Straftat nach dem GeschGehG dar.
Whistleblower haben jedoch keine Strafverfolgung zu befürchten, sofern sie unter anderem mit ihrem Hinweis rechtswidrige Handlungen aufdecken.\\
In dem Whistleblower stärker geschützt werden und Unternehmen, die diese benachteiligen sanktioniert werden, können Whistleblower vermehrt und ohne Konsequenzen Unternehmen und auch Behörden zur Verantwortung ziehen.
So wird die Transparenz und Integrität in der EU gefördert.
Auch der Rechenschaftspflicht und dem Vertrauen in europäische Organisationen ist die Richtlinie dienlich.
Es ergeben sich zwar einige Herausforderungen mit der neuen Richtlinie, diese lassen sich jedoch alle wie beschrieben bewältigen und stellen keine ernsthafte Gefahr für die Richtlinie und ihrer Wirksamkeit dar.\\
Insgesamt kann gesagt werden, dass die Whistleblowing-Richtlinie eine große Chance für mehr Gerechtigkeit in der Europäischen Union darstellt.