\documentclass[12pt, a4paper]{article}

%==================================================
%wichtige Packages
%==================================================

\usepackage[T1]{fontenc}                %Richtige Darstellung Sonderzeichen
\usepackage[utf8]{inputenc}             %Richtige Darstellung Umlaute
\usepackage[ngerman]{babel}             %Sprache: Deutsch
\usepackage{helvet}                     %Schriftart: Helvetica
\usepackage{setspace}   %Zeilenabstand: 1,5
\usepackage{import}                     %Import Datein
\usepackage{subfiles}                   %Unterkapitel
\usepackage{paralist}
\usepackage[hyphens]{url}               %Zeilenumbruch in URL
\usepackage[driver=pdftex]{geometry}    %Seitenränder
\usepackage[hang]{footmisc}             %Fußnote Strich unten
\usepackage{graphicx}                   %Für Bilder
\usepackage{longtable}                  %Abkürzungsverzeichnis
\usepackage{tocloft}                    %Anhangsverzeichnis
\usepackage{titlesec}                   %Anhangsverzeichnis
\usepackage{titletoc}                   %Anhangsverzeichnis
\usepackage{etoolbox}                   %Anhangsverzeichnis
\usepackage{csquotes}
\usepackage[
  style=authoryear, 
  urldate=long, 
  maxbibnames=99
  ]{biblatex}                             %Literaturverzeichnis
\usepackage[
  disable
  ]{todonotes}                            %Todos
\usepackage[hidelinks]{hyperref}        %hidelinks versteckt den roten Kasten
\usepackage{nameref}% loads gettitlestring
\usepackage{array}
\usepackage[labelfont={color=gray,small},textfont={color=gray,small}]{caption}
\usepackage{lineno}
\usepackage{multirow}
\usepackage{longtable}
\usepackage{pdfpages}
\usepackage{pgffor}
\usepackage{xparse}
\usepackage{lmodern,textcomp}
\usepackage{mathptmx}


%==================================================
%Literaturverzeichnis
%==================================================

\setstretch{1.15}
\setlength{\parindent}{0pt}

\addbibresource{Quellen.bib}

\DeclareDelimFormat{multinamedelim}{\addcomma\space}
\DeclareDelimFormat{finalnamedelim}{\addcomma\space}
\DeclareFieldFormat{title}{#1}
% \DeclareFieldFormat{url}{#1}
\DeclareFieldFormat{url}{\url{#1}}
\urlstyle{rm}
\DefineBibliographyStrings{german}{%
  andothers = {et\addabbrvspace al\adddot},
}
\DeclareFieldFormat{extradate}{{\mknumalph{#1}}}
\DeclareNameAlias{sortname}{family-given}

\AtBeginBibliography{%
  \renewcommand*{\multinamedelim}{\addspace\slash\addspace}%
  \renewcommand*{\finalnamedelim}{\addspace\slash\addspace}%
}

% Fußnotenformat
\renewbibmacro{cite}{\printtext[bibhyperref]{\printnames{labelname} (\iffieldundef{year}{o. J.}{\printfield{year}}\printfield{extradate})}}

% Formatierung des Literaturverzeichnisses
% Website
% \DeclareBibliographyDriver{online}{%
%   \textbf{\printnames{author} (\iffieldundef{year}{o. J.}{\printfield{year}}\printfield{extradate}):}\newline%
%   \textnormal{\printfield{title}}, \printfield{url}, \printfield{note}.\bigbreak}

\newcolumntype{R}[1]{>{\raggedright\arraybackslash}p{#1}}
\DeclareBibliographyDriver{online}{%
\begin{tabular}{R{5cm}R{10cm}}
    \textbf{\printnames{author}}  & \textnormal{\printfield{title}}, \printfield{url}, \printfield{note}. \\
\end{tabular}\bigbreak
}
% \textbf{(\iffieldundef{year}{o. J.}{\printfield{year}}\printfield{extradate})}

% Paper
% \DeclareBibliographyDriver{booklet}{%
%   \textbf{\printnames{author} (\iffieldundef{year}{o. J.}{\printfield{year}}\printfield{extradate}):}\newline%
%   \textnormal{\printfield{title}}.\bigbreak}
\DeclareBibliographyDriver{booklet}{%
\begin{tabular}{R{5cm}R{10cm}}
    \textbf{\printnames{author}}  & \textnormal{\printfield{title}}. \\
\end{tabular}\bigbreak
}

% Buch
% \DeclareBibliographyDriver{book}{%
%   \textbf{\printnames{author} (\iffieldundef{year}{o. J.}{\printfield{year}}\printfield{extradate}):}\newline%
%   \textnormal{\printfield{title}}, \iffieldundef{edition}{o. A.}{\printfield{edition}}, \printlist{publisher}\iflistundef{location}{}{, \printlist{location}}.\bigbreak}     
\DeclareBibliographyDriver{book}{%
\begin{tabular}{R{5cm}R{10cm}}
    \textbf{\printnames{author}}  & \textnormal{\printfield{title}}, \iffieldundef{edition}{o. A.}{\printfield{edition}}, \printlist{publisher}\iflistundef{location}{}{, \printlist{location}}. \\
\end{tabular}\bigbreak
}

%==================================================

% \paragraph and \subparagraph for forth and fifth level headings

\setcounter{secnumdepth}{5}
\setcounter{tocdepth}{4}

\titleformat{\paragraph}
    {\normalfont\normalsize\bfseries}{\theparagraph}{1em}{}
\titlespacing*{\paragraph}
    {0pt}{1.5ex plus 1ex minus .2ex}{0.5ex plus .2ex}

\titleformat{\subparagraph}
    {\normalfont\normalsize\bfseries}{\thesubparagraph}{1em}{}
\titlespacing*{\subparagraph}
    {0pt}{3.25ex plus 1ex minus .2ex}{1.5ex plus .2ex}



      \newcommand{\ipone}[3]{\footnote{
        #1 \hyperref[#2]{Anhang \ref*{#2}, Z. #3}.
      }}

      \newcommand{\iptwo}[3]{\footnote{
        #1 \hyperref[#2]{Anhang \ref*{#2}, Z. #3}.
      }}

      \newcommand{\iponetwo}[5]{\footnote{
        #1 \hyperref[#2]{Anhang \ref*{#2}, Z. #3}; \hyperref[#4]{Anhang \ref*{#4}, Z. #5}.
      }}

      \NewDocumentCommand{\citeappendix}{O{} O{} m}{#1 \hyperref[#3]{Anhang \ref*{#3}, S. \pageref*{#3}}#2}

      \NewDocumentCommand{\footappendix}{O{} O{} m}{\footnote{\citeappendix[#1][#2]{#3}.}}
      
      \NewDocumentCommand{\footappendencies}{O{} O{} m O{} m}{\footnote{
          #1 \hyperref[#3]{Anhang \ref*{#3}, S. \pageref*{#3}}#2; \hyperref[#5]{Anhang \ref*{#5}, S. \pageref*{#5}}#4.
        }}
      
       



%==================================================
%Einstellung Seitenränder
%==================================================
\geometry{
        left=5.25cm,
        right=2cm,
        top=2cm,
        bottom=2cm
    }
%==================================================

%==================================================
% Todos
%==================================================
\newcommand\todoin[2][]{\todo[inline, caption={2do}, #1]{
 \begin{minipage}{\textwidth-4pt}#2\end{minipage}}}

% Todos to left side
\reversemarginpar

% Todos Width
\setlength{\marginparwidth}{3.5cm}

% Todos Color
\definecolor{NOTES}{HTML}{00ffff}
\definecolor{REVIEW}{HTML}{d400ff}
\definecolor{TODO}{HTML}{90EE90}
%==================================================

%Variable zum speichern der Seite
\newcounter{savepage}

\begin{document}

    \newcounter{romanSection} % Inhaltsverzeichnis, etc, Mit Anhang
    \newcounter{romanPage}    % Inhaltsverzeichnis, etc, ohne Anhang


    %==================================================
    %Deckblatt + Sperrvermerk
    %==================================================

    {\pagestyle{empty}

    %=============================================================
%Deckblatt
%=============================================================

\noindent
\begin{center}
    \vspace*{9cm}
    \textbf{Whistleblowing}
    \bigbreak
    \bigbreak
    \bigbreak
    \textbf{Proseminararbeit}
    \bigbreak
    \bigbreak
    \textbf{Dr. iur. Malte Wilke, LL.M.}
    \bigbreak
    \bigbreak
    \textbf{Sommersemster 2024}
    \bigbreak
    \bigbreak
    \textbf{10044388}
\end{center}

\newpage

    }
    \newpage

    {\pagestyle{empty}

    %=============================================================
%Eigenständigkeitserklärung
%=============================================================
\thispagestyle{empty}

\noindent
\begin{LARGE}
    \begin{center}
        \textbf{Eigenständigkeitserklärung} \\
    \end{center}
\end{LARGE}
    
\bigbreak
\bigbreak

\noindent
Hier die Erklärung.

    }
    \newpage


    %==================================================
    %Inhaltlsverzeichnis
    %==================================================
    
    {   
        %Kapitel und Seitennummerierung wird auf Römisch 1 gesetzt
        \setcounter{page}{1}
        \renewcommand{\thesection}{\Roman{section}}
        \pagenumbering{Roman}

    \addtocontents{toc}{\protect\setcounter{tocdepth}{0}}
    %Übershrift: Inhaltsverzeichnis
    \section{Inhaltsverzeichnis}        

    %Befehl, damit Inhaltsverzeichnis nicht doppelt da steht 
    \renewcommand{\contentsname}{}

    %Befehle zur Strukturierung des Inhaltsverzeichnisses (Anordnung)
    \setlength\cftbeforesecskip{3pt}
    \renewcommand{\cftsecleader}{\cftdotfill{\cftdotsep}}

    %Inhaltsverzeichnis wird erzeugt
    \tableofcontents 

    \newpage


    %==================================================
    %Abkürzungsverzeichnis
    %==================================================
    
   


    \section{Literaturverzeichnis}

\begin{longtable}{R{5 cm}R{8,5 cm}}

    \textbf{Altenbach, Thomas/ Dierkes, Kevin} & Corporate Compliance Zeitschrift, 2020, S. 126 ff, „EU-Whistleblowing-Richtlinie und DSGVO“ (zit.:Altenbach/Dierkes, CCZ 2020, 126.) \\
    \\

    \textbf{Bayreuther, Frank} & Neue Zeitschrift für Arbeitsrecht - Beilage, 2022, S. 20 ff,  „Whistleblowing und das neue Hinweisgeberschutzgesetz“ (zit.: Bayreuther, NZA-Beilage 2022, 20.) \\
    \\

    \textbf{Bendel, Oliver} & Gabler Wirtschaftslexikon, „Whistleblowing“, Fundort: https://wirtschaftslexikon.gabler.de/definition/
    
    whistleblowing-53526, Abrufdatum: 22.05.2024  (zit.: Bendel, Whistleblowing.) \\
    \\

    \textbf{Brockhaus, Robert} & Zeitschrift für Internationale Strafrechtswissenschaft, 2020, Ausgabe 3, S. 102 ff, „Das Geschäftsgeheimnisgesetz - Zur Frage der Strafbarkeit von Hinweisgebern unter Berücksichtigung der Whistleblowing-Richtlinie“ (zit.: Brockhaus, ZIS 3/2020, S.) \\
    \\

    \textbf{Dilling, Johannes} &  Corporate Compliance Zeitschrift, 2019, S. 214 ff, „Der Schutz von Hinweisgebern und betroffenen Personen nach der EU-Whistleblower-Richtlinie“ (zit.: Dilling, CCZ 2019, 214.) \\
    \\

    \textbf{Degenhart, Maximilian/ Dziuba, Anne} & Betriebs Berater, 9. Auflage, 01.03.2012: „Die EU-Whistleblower-Richtlinie und ihr earbeitsrechtlichenA uswirkungen“ (zit.: Degenhart/Dziuba, BB 2021/9, S.) \\
    \\

    \textbf{Dzida, Boris/ Granetzny, Thomas} & Neue Zeitschrift für Arbeitsrecht, 2020, S. 1201 ff, „Die neue EU-Whistleblowing-Richtline und ihre Auswirkungen auf Unternehmen“ (zit.: Dzida/Granetzny, NZA 2020, 1201.) \\
    \\ 

    \textbf{Dzida, Boris/ Seibt, Christopj} & Neue Zeitschrift für Arbeitsrecht, 2023, S. 657 ff, „Neues Hinweisgeberschutzgesetz: Analyse und Antworten auf Praxisfragen“ (zit.: Dzida/Seibt, NZA 2023, 657.) \\
    \\

    \textbf{Europäische Kommission} & Estimating the economic benefits of whistleblower protection in public procurement : final report, Publications Office, 2017, https://data.europa.eu/doi/10.2873/125033 (zit.: Studie der Europäischen Kommission 2017, S.) \\
    \\
    
    \textbf{Gerdemann, Simon} & Neue Zeitschrift für Arbeitsrecht Beilage, 2020, S. 43 ff, „Whistleblower als Agenten des Europarechts“ (zit.: Gerdemann,NZA-Beilage 2020, 43.) \\
    \\

    \textbf{Gerdemann, Simon} & Soziales Recht, Vol. 11, No. 1, Februar 2021 „Whistleblowing, quo vadis? Die Europäische Whistleblower-Richtlinie und ihre Umsetzung in deutsches Recht – Teil 1: Die Richtlinie und ihre neuralgischen Punkte“ (zit.: Gerdemann, SR 1, 2021) \\
    \\

    \textbf{Martinek, Micheal/ Semler, Franz-Jörg/ Flohr, Eckhard} & Handbuch des Vertriebsrechts, 4. Auflage, München 2016 (zit.: Handbuch VertriebsR, \textit{Bearbeiter}, § Rn.) \\
    \\

    \textbf{Meyer-Krumenacker, Astrid} & QZ-online.de „Risiken des Hinweisgeberschutzgesetzes“, 02.11.2023 Fundort: https://www.qz-online.de/a/fachartikel/risiken-des-hinweisgeberschutzgesetzes-5186581, Abrufdatum: 28.05.2024 (zit.: Meyer-Krumenacker, Risiken des Hinweisgeberschutzgesetzes.) \\
    \\

    \textbf{Ohly, Ansgar} & Gewerblicher Rechtsschutz und Urheberrecht, Ausgabe 5, 2019: „Das neue Geschäftsgeheimnisgesetz im Überblick“ (zit.: Ohly, GRUR 2019, 441.) \\
    \\

    \textbf{Paal, Boris P./ Pauly, Daniel A.} & Beck'sche Kompakt-Kommentare Datenschutz-Grundverordnung, 3. Auflage, München 2021 (zit.: Paal/Pauly DSGVO,\textit{Bearbeiter}, Art. Rn.) \\
    \\

    \textbf{Reinbacher, Tobias} & Kriminalpolitische Zeitschrift, Ausgabe 3, 2019: „Der neue Straftatbestand des § 23 GeschGehG und das Whistleblowing“ (zit.: Reinbacher, KriPoZ 3/2019, S.) \\
    \\

    \textbf{Rohrlich, Michael} & lexware „Hinweisgeberschutzgesetz – was ist das und welche Maßnahmen müssen Unternehmen umsetzen?“, 16.05.2024, Fundort: https://www.lexware.de/wissen/unternehmensfuehrung/
    
    hinweisgeberschutzgesetz-das-muessen-unternehmen-jetzt-tun/, Abrufdatum: 28.05.2024 (zit.: Rohrlich, Hinweisgeberschutzgesetz.) \\
    \\

    \textbf{Schürrle, Thomas/ Fleck, Franziska} & TCorporate Compliance Zeitschrift, 2011, S. 218 ff, „„Whistleblowing Unlimited” – Der U.S. Dodd-Frank Act und die neuen Regeln der SEC zum Whistleblowing” (zit.: Schürrle/Fleck, CCZ 2011, 218.) \\
    \\

    \textbf{Schmolke, Klaus Ulrich} & Neue Zeitschrift für Gesellschaftsrecht, 2020, S. 5 ff, „Die neue Whistleblower- Richtlinie ist da! Und nun?“ (zit.: Schmolke, NZG 2020, 5) \\
    \\

    \textbf{Taschke, Jürgen/ Pielow, Tobias/ Volk, Ela} & Neue Zeitschrift für Wirtschafts-, Steuer- und Unternehmensstrafrecht, 2021, S. 85 ff, „Die EU-Whistleblower-Richtlienie - Herausfprderungen für die Unternehmenspraxis“ (zit.: Taschke/Pielow/Volk, NZWiSt 2021, 85.) \\
    \\

    \textbf{Thüsing, Gregor/ Rombey, Sebastian} & Neue Zeitschrift für Gesellschaftsrecht, 2018, S. 1001 ff, „Nachdenken über den Richtlinienvorschlag der EU-Kommission zum Schutz von Whistleblowern“ (zit.: Thüsing/Rombey, NZG 2018, 1001.) \\
    \\

    \textbf{Yakimova, Yasmina} & Whistleblower: Neue Vorschriften für EU-weiten Schutz von Informationen, Pressemitteilung Europäisches Parlament vom 16.04.2029; Fundort: https://www.europarl.europa.eu/news/de/press-room/20190410IPR37529/whistleblower-neue-vorschriften-fur-eu-weiten-schutz-von-informanten; Abrufdatum: 27.05.2024 (zit.: Yakimova, Whistleblower: Neue Vorschriften für EU-weiten Schutz von Informationen.) \\
    \\

    \textbf{Weber, Klaus} & Rechtswörterbuch, 32. Edition, München 2024 (zit.: Rechtswörterbuch, \textit{Bearbeiter}, Eintrag.) \\
    \\
    
\end{longtable}

    \newpage
    

    %==================================================
    %Abbildungsverzeichnis
    %==================================================

    %Überschrift: Abbildungsverzeichnis
    % \section{Abbildungsverzeichnis}

    % %Befehl, damit Abbildungsverzeichnis nicht doppelt da steht 
    % \renewcommand{\listfigurename}{}

    % %Abbildungsverzeichnis wird erzeugt
    % \listoffigures                      

    % }

    % KapitelNummerierung merken
    % \setcounter{romanSection}{\thesection}
    % \setcounter{romanPage}{\arabic{page}}


    % \newpage

    
    %==================================================
    %Inhaltliches
    %==================================================

    {
        % KapitelNummerierung auf 0 setzen und mit arabischen Zahlen belegen
        % \setcounter{section}{0}
        \setcounter{page}{1}
            \pagenumbering{arabic}

    %Hier steht der Inalt
    \newcounter{mainsection}
    % \setcounter{mainsection}{0}
    \renewcommand{\thesection}{\Alph{mainsection}.}
    \renewcommand{\thesubsection}{\Roman{subsection}.}
    \renewcommand{\thesubsubsection}{\arabic{subsubsection}.}
    \renewcommand{\theparagraph}{\alph{paragraph})}

    \let\oldsection\section % Erstellen Sie eine Kopie des ursprünglichen \section Befehls

    \renewcommand{\section}[1]{ % Überschreiben Sie den \section Befehl
      \stepcounter{mainsection} % Inkrementieren Sie den Zähler
      \oldsection{#1} % Rufen Sie den ursprünglichen \section Befehl auf
    }

    \addtocontents{toc}{\protect\setcounter{tocdepth}{4}}

    \section{Einleitung}
\label{sec:Einleitung}

\todo[color=TODO]{Die Einleitung schreiben}
Hier steht die Einleitung.
    
    \section{Whistleblowing}
\label{sec:Hauptteil}

\subsection{Die EU Whistleblower Richtline}
\label{sec:Teilkapitel}

\subsubsection{Definition Whistleblowing}
Als Whistleblower wird eine Person bezeichnet, die Informationen über Missstände aus dem Innenbereich einer Organisation im öffentlichen Interesse einer Behörde bzw.  der breiteren Öffentlichekit (exterenes Whistleblowing) oder einer organisationsinternen Stelle (internes Whistleblowing) mitteilt.\footnote{Rechtswörterbuch, \textit{Kallos}, Whistleblower.} 
Meist handelt es sich bei einem Whistleblower um einen (etablierten oder ehemaligen) Mitarbeiter oder einen Kunden, welcher aus eigener Erfahrung berichtet.\footnote{Bendel, Whistleblowing.}
Das Whistleblowing ist von der bloßen Beschwerde über persönliche Umstände abzugrenzen, es setzt einen Umstand von allgemeinem Interesse voraus.\footnote{Bendel, Whistleblowing.}


\subsubsection{Aussagen der EU-Whistleblower Richtline}

Die Richtlinie (EU) 2019/1937 des Europäischen Parlaments und des Rates vom 23. Oktober 2019 zum Schutz von Personen, die Verstöße gegen das Unionsrecht melden, auch Whistleblowing-Richtlinie genannt, ist am 23.10.2019 erlassen worden.\footnote{Taschke/Pielow/Volk, NZWiSt 2021, 85.}
Sie wurde am 31.05.2023 durch das Hinweisgeberschutzgesetz (HinSchG) in deutsches Recht umgesetzt.
Durch die Richtlinie werden zwei Hauptziele verfolgt: zum einen sollen Whistleblower eine klar reglementierte Möglichkeit zur Meldung von Missständen haben und zum anderen sollen sie vor Repressialien geschützt werden.\footnote{Dzida/Granetzny, NZA 2020, 1201.}\\
Nun wird genauer auf die Einzelnen Aussagen der Richtlinie eingegangen.

\paragraph{Anwendungsbereich der Richtlinie}
Damit Whistleblower von dem Schutz durch die Richtline profitieren, müssen sie den sachlichen sowie den persönlichen Anwendungsbereich erfüllen.\footnote{Taschke/Pielow/Volk, NZWiSt 2021, 85.}\\
Der sachliche Anwendungsbereich wird in Art. 2 I der Whistleblower-Richtlinie (WBRL) geregelt. 
Der Anwendungsbereich begrenzt sich abschließend auf die dort aufgeführten Verstöße gegen Unionsrecht.\footnote{Taschke/Pielow/Volk, NZWiSt 2021, 85.}
Zum Beispiel sind dies Verstöße in den Bereichen öffentliche Auftragsvergabe, Finanzdienstleistungen, Geldwäsche und Terrorismusfinanzierung, Produktsicherheit, Verkehrssicherheit, Umweltschutz, kerntechnische Sicherheit, Lebensmittel- und Futtermittelsicherheit, Tiergesundheit und Tierschutz, öffentliche Gesundheit, Verbraucherschutz, Schutz der Privatsphäre, Datenschutz, Sicherheit von Netz- und Informationssystemen, EU-Wettbewerbsvorschriften, Körperschaftssteuervorschriften sowie Verstöße gegen die finanziellen Interessen der EU.\footnote{Dzida/Granetzny, NZA 2020, 1201.}
So fällt bloßes unethisches aber nicht rechtswidriges Fehlverhalten nicht in den Anwendungsbereich der Richtlinie.\footnote{Taschke/Pielow/Volk, NZWiSt 2021, 85.}
Ein größerer Anwendungsbereich war aufgrund der beschränkten Gesetzgebungskompetenz der EU nicht möglich, nach Art. 2 II WBLR bleibt es den Mitgliedstaaten jedoch freigestellt den Schutz in Bezug auf Bereiche oder Rechtsakte auszudehnen, die nicht unter Absatz 1 fallen.\footnote{Dzida/Granetzny, NZA 2020, 1201.}
Der deutsche Gesetzgeber ergänzte die Richlinie durch § 2 I Nr. 1, 2 und 10 HinSchG zum einen um „Verstöße, die strafbewehrt sind“, und zum anderen um „Verstöße, die bußgeldbewehrt sind, soweit die verletzte Vorschrift dem Schutz von Leben, Leib oder Gesundheit oder dem Schutz der Rechte von Beschäftigten oder ihrer Vertretungsorgane dient“.\footnote{Dzida/Seibt, NZA 2023, 657.}

\bigbreak

Der persönliche Anwendungsbereich wird in Art. 4 WBRL geregelt.
Aus den Wortlaut ergibt sich, dass der Hinweisgeber „im privaten oder im öffentlichen Sektor tätig“ sein und „im beruflichen Kontext Informationen über Verstöße erlangt haben“ muss.
Es sind demnach im Grunde alle Personen geschützt, die im weitesten Sinne in einer “arbeitsbezogenen Verbinung“ mit dem Unternehmen stehen.\footnote{Dzida/Granetzny, NZA 2020, 1201.}
Das Motiv des Hinweisgebers spielt keine Rolle.\footnote{Schmolke, NZG 2020, 5.}


\paragraph{Voraussetzungen für den Schutz}
In Art. 6 der Whistleblower-Richtlinie wird geregelt, welche Voraussetzungen für den Schutz erfüllt sein müssen.
Hiernach hat ein Hinweisgeber für einen Anspruch auf Schutz, sofern er im Zeitpunkt der Meldung im Hinblick auf die Umstände und die verfügbaren Informationen mit hinreichendem Grund davon ausgehen durfte, dass die gemeldete Information zutrifft und einen Verstoß darstellt, der in den Anwendungsbereich der Richtlinie fällt, vgl. Art. 6 I a)WBRL.\footnote{Taschke/Pielow/Volk, NZWiSt 2021, 85.}
Der Hinweisgeber muss demanch gutgläubig sein, wer also bewusst falsche oder irreführende Informationen meldet ist nicht schutzbedürftig.\footnote{Degenhart/Dziuba, BB 2021/9, 573.}\\
Die Meldung kann zunächst intern, also an eine geeignete Stelle inerhalb des Unternehmens, oder extern an die zuständige Behörde geschehen, vgl. Art 6 I b) WBRL i.V.m. Art. 7, Art. 10 WBRL.\footnote{Schmolke, NZG 2020, 5.}
Die interne und die externe Meldung stehen gleichrangig nebeneinander.\footnote{Dzida/Granetzny, NZA 2020, 1201.}
Höhere Anforderung an den Schutz durch die Richtlinie werden dann gestellt, wenn der Hinweisgeber sich direkt an die Öffentlichekit wendet, vgl. Art. 15 WBRL.\footnote{Schmolke, NZG 2020, 5.}
So muss der Hinweisgeber entweder den Verstoß zunächst intern oder extern gemeldet haben, ohne dass innerhalb der in Art. 11 WBRL festgelegten Höchstfristen von drei bzw. sechs Monaten geeignete Maßnahmen ergriffen worden sind.
Andere Situationen in denen der Hinweisgeber sich direkt an die Öffentlichekit wenden kann sind wenn er davon ausgehen durfte, dass der Verstoß eine „unmittelbare oder offenkundige Gefährdung des öffentlichen Interesses“ darstellt, oder aufgrund der besonderen Umstände des Falls geringe Aussichten bestehen, dass wirksam gegen den Verstoß vorgegangen wird, etwa bei Kollusion zwischen dem Delinquenten und der Behörde (Art. 6 I Buchst. b iVm Art. 15 I WBRL).\footnote{Schmolke, NZG 2020, 5.}
Die Meldung an die Öffentlichekit ist mithin als \textit{ultima ratio} subsidiär\footnote{Dzida/Granetzny, NZA 2020, 1201.} und der Hinweisgeber sollte sich zunächst an die eingerichteten internen oder externen Stellen richten.

\paragraph{Schutzmaßnahmen}
Sind die sachlichen und persönlichen Voraussetzungen erfüllt, wird der Schutz der Hinweisgeber vor allem durch das Verbot jeglicher Repressalien aus Art. 19 ff WBRL gewährleistet.\footnote{Dzida/Granetzny, NZA 2020, 1201.}
Eingeschlossen hierin ist ein Verbot für die Mitgliedstaaten Repressalien nur anzudrohen oder zu versuchen, vgl. Art. 19 WBRL.
Art. 19 WBRL stellt zur besseren Einschätzung einen Katalog an Regelbeispielen für mögliche Repressalien zur Verfügung.
Beispiele hieraus sind unter anderem Suspendierung oder Kündigung, Herabstufung oder Versagung einer Beförderung, Aufgabenverlagerung, negative Leistungsbeurteilung, sowie Nötigung und Diskriminierung.\\
Dieses Verbot wird durch die in Art. 21 V WBRL geregelte Beweislastumkehr verstärkt.
Hiernach muss nun das Unternehmen beweisen, dass es sich bei einer arbeitsrechtlichen Maßnahme nicht um eine unzulässige Repressialie i.S.d. Art. 19 WBRL handelt.\footnote{Dzida/Granetzny, NZA 2020, 1201.}
Somit ist nicht mehr der Arbeitnehmer in der Verantwortung darzulegen, dass eine arbeitsrechtliche Maßnahme aufgrund des Whistleblowings erfolgt ist.
Dies soll potentiellen Beweisschwierigkeiten, die sich für den Hinweisgeber ergeben könnten, entgegen wirken.\footnote{Taschke/Pielow/Volk, NZWiSt 2021, 85.}
Das Verbot jeglicher Repressalien wird ferner durch Art. 23 WBRL untermauert, wonach die Mitgliedstaaten “wirksame, angemessene und abschreckende“ Sanktionen für juristische sowie natürliche Personen festlegen müssen, die Meldungen behindern, Repressalien ergreifen, mutwillig Gerichtsverfahren gegen Hinweisgeber oder geschützte Dritte anstregen oder gegen das Vertraulichkeitsgebot im Rahmen des Meldeverfahrens verstoßen.\footnote{Schmolke, NZG 2020, 5.}
Somit wird sichergestellt, dass Hinweisgeber nach ihrer Meldung keine Konsequenzen zu befürchten haben.

\paragraph{Meldewege}
Potentiellen Hinweisgeber sollen nicht bloß geschützt werden, ihnen soll auch eine effektive Meldeinfrastruktur zur Verfügung gestellt werden.\footnote{Schmolke, NZG 2020, 5.}
Es wird zwischen internen und externen Meldekanälen unterschieden.\footnote{Schmolke, NZG 2020, 5.}
Art. 8 III WBRL gibt vor, dass Unternehmen mit 50 oder mehr Arbeitnehmern einen internen Meldekanal nach Art. 8 I errichten müssen.\footnote{Taschke/Pielow/Volk, NZWiSt 2021, 85.}
Diese müssen nach Art. 9 I a) WBRL zur Gewährleistung der Vertraulichkeit der Identität des Hinweisgebers und erwähnter Personen entsprechend sicher konzipirt, implementiert und betrieben werden.\footnote{Taschke/Pielow/Volk, NZWiSt 2021, 85.}
Zudem muss das Unternehmen dem Hinweisgeber nach Art. 9 I b) WBRL innerhalb von sieben Tagen den Eingang der Meldung bestätigen und ihm nach spätestens drei Monaten berichten, wie mit seinem Hinweis umgegangen wurde und was für Maßnahmen ergriffen worden sind.\footnote{Dzida/Granetzny, NZA 2020, 1201.}
Es wird somit sichergesetellt, dass den Meldungen der Hinweisgeber auch nachgegangen wird.\\
Für die externen Meldekanäle müssen gemäß Art. 10 WBRL durch die Mitgliedstaaten zuständige Behörden benannt werden, die befugt sind Meldungen entgegenzunehmen, Rückmeldung zu geben und entsprechende Folgemaßnahmen zu ergreifen.\footnote{Dzida/Granetzny, NZA 2020, 1201.}
Zu diesem Zweck müssen die Behörden mit angemessenen Ressourcen ausgestattet werden, vgl. Art. 11 I WBRL.
Zur Wahrung des Vertraulichkeitsgebots müssen die Meldekanäle grundsätzlich so konzipirt sein, dass Unbefugte keinen Zugriff auf die Daten erhalten können, vgl. Art 16 WBRL.\footnote{Dzida/Granetzny, NZA 2020, 1201.}



\subsubsection{Beeinflussung der Verabschiedung durch Entwicklungen in den USA}

Das Whistleblowing-Recht in den USA hat dort eine langjährige Tradition.\footnote{Gerdemann, NZA-Beilage 2020, 43.}
Hintergrund ist das Bedürfnis des dortigen Bundesgesetzgebers, trotz der vergleichsweise beschränkten eigenen Administrativressourcen, Verstöße gegen bundesrechtliche Normen in Erfahrung zu bringen und hierdurch eine wirksame praktische Durchsetzung seiner seit dem letzten Jahrhundert sukzessive expandierenden Legislativkompetenzen zu erwirken.\footnote{Gerdemann, NZA-Beilage 2020, 43.}
Dieser Regulierungsmechanismus der USA führte zunehmend zu Erfolgen, sodass sich die Europäische Union das Prinzip des Whistleblowings zu eigen gemacht hat.\footnote{Gerdemann, NZA-Beilage 2020, 43.}
Die USA fungierte mithin als legislatives Vorbild.
Vor allem zwei Akte zogen die Aufmerksamkeit der Europäischen Unions auf sich: Der Sarbanes-Oxley Act von 2002 (SOX) und der Dodd-Frank Act von 2010.\footnote{Gerdemann, SR 1, 2021}\\
Der erst genannte ist ein vom amerikanischen Kongress verabschiedetes Bundesgesetz mit dem Ziel, das Vertrauen der Anleger in die Richtigkeit und Verlässlichkeit der veröffentlichten Finanzdaten von Unternehmen wiederherzuestellen.\footnote{CC, \textit{Obermayr}, § 44 Rn. 111.}
Er wurde als Reaktion auf die Bilanzskandale erlassen.\footnote{Schürrle/Fleck,CCZ 2011, 218.}
Unter anderem wird den betroffenen Unternehmen die Pflicht zur Errichtung eines internen Kontrolsystems sowie die Etablierung eines Beschwerdemanagements (Whistleblower-System) auferlegt.\footnote{Handbuch VertriebsR, \textit{Passarge}, § 79 Rn. 120.}
Von dem Act betroffen sind alle Unternehmen, deren Aktien an der New York Stock Exchange gehandelt werden; hierunter sind auch 23 deutsche Unternehmen.\footnote{CC, \textit{Obermayr}, § 44 Rn. 111.}
Mithin ist der SOX grundsätzlich unabhängig vom Sitz des Unternehmens anwendbar\footnote{Handbuch VertriebsR, \textit{Passarge}, § 79 Rn. 119.}, hierdurch ergibt sich bereits eine Relevanz für die EU.
Bereits im SOX wurden Hinweisgeber vor Diskriminierung in Form von Entlassung, Zurückstufung oder ähnlichem geschützt, bei Verstoß konnten den Arbeitnehmern Schadenersatzansprüche zustehen.\footnote{Schürrle/Fleck,CCZ 2011, 218.}\\
Der Dodd-Frank Act wurde schließlich als Reaktion auf den Finanzskandal 2008 erlassen.\footnote{Schürrle/Fleck,CCZ 2011, 218.}
Durch diesen Act sollte das Antidiskriminierungsrecht des SOX noch verbessert werden und zudem wurde mit dem „SEC Office of the Whistleblower“ eine externe Meldebehörde etabliert.\footnote{Gerdemann, SR 1, 2021}
Durch den Dodd-Frank Act wurde also der Schutz der Hinweisgeber vor Vergeltung ausgeweitet, hierfür wurden Durchführungsbestimmungen vorgegeben, zudem wurde der Anwendungsbereich auf nicht-börsengelistete Tochterunternehmen ausgeweitet.\footnote{Schürrle/Fleck,CCZ 2011, 218.}
Die Europäische Union sah sich mit ähnlichne Herausforderungen wie das Federal Government der USA konfrontiert.\footnote{Gerdemann, SR 1, 2021}
So ist es nicht verwunderlich, dass die Whistleblowing-Richtlinie der Europäischen Union den Regelungen aus dem USA zum Teil ähnelt.\\
Zudem kommt der Faktor, dass durch internationale Whistleblowing-Fälle das öffentliche Interesse am Phänomen des Whistleblowings gestiegen ist.\footnote{Gerdemann, SR 1, 2021} 
Beispiele für bekannte Fälle wären unter anderem Edward Snowden, Luxemburg-Leaks oder die Panama Papers.
Durch diese wurde verstärkt deutlich, dass Hinweisgeber eine wichtige Rolle bei der Aufedeckung von Verstößen gegen das EU-Recht spielen können, die das öffentliche Interesse und das Wohlergehen der Bürger und der Gesellschaft schädigen.\footnote{Yakimova, Whistleblower: Neue Vorschriften für EU-weiten Schutz von Informationen.}
Trotz der großen unionsrechtlichen Relevanz hatten vor Erlass der Richtlinie nur 10 EU-Länder einen umfassenden Schutz für Hinweisgeber.\footnote{Yakimova, Whistleblower: Neue Vorschriften für EU-weiten Schutz von Informationen.}
Eine breit angelegte, sektorübergreifende Regulierung des Whistleblowings in Europa schien dennoch zunächst eher unwahrscheinlich, bis sich eine Koalition aus staatlichem Rechtsdurchsetzungsinteresse und zivilgesellschaftlichem Engagement bildete.\footnote{Gerdemann, SR 1, 2021} 

\subsection{Mit dem Whistleblowing verbunden Herausforderungen}
Die Erlassung der Whistleblowing-Richtlinie bewirkt einen wichtigen Beitrag zum Schutz von Hinweisgebern. Die tatsächliche Umsetzung ist aber nicht ohne Herausforderungen möglich.

\subsubsection{Arbeitsrechtliche Herausforderungen}
Zunächst erwachsen im Arbeitsrecht einige Herausforderungen für die Arbeitgeber.\footnote{Degenhart/Dziuba, BB 2021/9, 571.}
Juristische Personen des Privatrechts werden zur Errichtung eines Hinweisgebersystems verpflichtet.
Auch kleine Unternehmen mit 50 oder mehr Mitarbeitern müssen seit dem 17. Dezember 2023 eine interne Meldestelle eingerichtet haben, wobei sie zusammen mit anderen Unternehmen eine “gemeinsame Meldestelle“ betreiben können, vgl. Art. 8 III WBRL/§ 12 HinSchG.\footnote{Rohrlich, Hinweisgeberschutzgesetz.}
Für die regelmäßige Kontrolle und Bearbeitung der eingehenden Meldungen muss dementsprechend eine zuständige Person oder Stelle benannt werden.\footnote{Rohrlich, Hinweisgeberschutzgesetz.}
Hierfür müssen nicht unerhebliche personelle Ressourcen aufgewendet werden.\\
Die Meldekanäle müssen zudem nach Art. 9 WBRL einige Anforderungen erfüllen. 
Unter anderem müssen diese so sicher konzipiert, eingerichtet und betrieben werden, dass die Vertraulichkeit der Identität des Hinweisgebers gewahrt bleibt, weiter muss dem Hinweisgeber der Eingang seiner Meldung innerhalb von sieben Tagen bestätigt werden und es muss eine Rückmeldung innerhalb von drei Monaten erfolgen.\footnote{Degenhart/Dziuba, BB 2021/9, 572.}
Sollte dies nicht gewährleistet sein, drohen den Unternehmen eine Geldbuße von bis zu 20.000 Euro, da eine Verletzung als Ordnungswidrigkeit gewertet wird.\footnote{Rohrlich, Hinweisgeberschutzgesetz.}\\
Ein weiterer Punkt, der arbeitsrechtlich zu Herausforderungen führen könnte, könnte die Beweislastumkehr aus Art. 21 V WBRL sein.
Dieses System kann potenziell von Arbeitnehmern ausgenutzt werden, indem diese sich durch einen rechtzeitigen „Hinweis“ zusätzlichen Kündigungsschutz verschaffen.\footnote{Dzida/Granetzny, NZA 2020, 1201.}
Der Arbeitgeber müsste nun erst einmal nachweisen, dass er den Arbeitnehmer nicht aufgrund des Hinweises kündigen wollte.
Dies ist vor allem bei wie der Probezeitkündigung, in denen eine rechtliche Notwendigkeit zur Begründung einer Maßnahme normalerweise gar nicht besteht, relevant.\footnote{Degenhart/Dziuba, BB 2021/9, 573.}
Der Arbeitgeber muss nun faktisch doch begründen, warum die Kündigung stattfand, um zu beiweisen, dass sie nicht auf dem Hinweis beruhte.\footnote{Degenhart/Dziuba, BB 2021/9, 573.}
Es ist zur Verhinderung eine umfassende Dokumentation der Unternehmen von zum Beispiel Mitarbeiterbewertungen, Karriereentwicklungen, erteilte Abmahnungen sowie bereits aufgetretene Probleme, notwendig, sodass gegebenenfalls im Kündigungsprozess nachgewiesen werden kann, dass dieser nicht mit dem vermeintlichen Whistleblowing zusammenhängt.\footnote{Dzida/Granetzny, NZA 2020, 1201.}
Mit dieser umfassenden Dokumentation ist aber eben auch ein enormer Aufwand verbunden, den die Arbeitgeber nun zusätzlich zu bewältigen haben.



\subsubsection{Datenrechtliche Herausforderungen}
Auch in Verbindung mit der Datenschutzgrundverordnung (DSGVO) kann es zu Spannungen kommen.
Gem. Art. 1 DSGVO zielt diese Verordnung darauf ab natürliche Personen bei der Verarbeitung personenbezogener Daten zu schützen.
Ferner sollen die Grundrechte und Grundfreiheiten natürlicher Personen und insbesondere deren Recht auf Schutz personenbezogener Daten geschützt werden.
Für die Verarbeitung personenbezogener Daten ist immer eine Rechtsgrundlage erforderlich.\footnote{Altenbach/Dierkes, CCZ 2020, 126.}
Bislang ergab sich diese in Bezug auf Hinweisgebersysteme gem. Art. 6 I 1 c) DSGVO nur für Unternehmen in ausgewählten Wirtschaftsbereichen.
Durch die Umsetzung der Whistleblowing-Richtlinie ergibt sich aus der nationalen Umsetzungsnorm zu Art. 8 WBRL eine entsprechende Rechtsgrundlage.\footnote{Altenbach/Dierkes, CCZ 2020, 126.}
In Deutschland ergibt sich diese aus § 12 HinSchG.
Hiervon können jedoch nur Unternehmen gebrauch machen, welche in den Anwendungsbereich der Whistleblowing-Richtlinie fallen.
Somit müssen vor allem Unternehmen mit weniger als 50 Mitarbeitern weiterhin gemäß Art. 6 I 1 f) DSGVO abwägen, ob die Verarbeitung zur Wahrung der berechtigten Interessen des Unternehmens erforderlich ist und die Interessen oder die Grundrechte der betreffenden Person nicht überwiegt.\footnote{Altenbach/Dierkes, CCZ 2020, 126.}\\
Ferner könnte es zu Konflikten mit Art. 14 DSGVO kommen.
In Art. 14 DSGVO wird die Art und der Umfang der Informationspflicht des Verantwortlichen gegenüber der betroffenen Person geregelt, wenn und soweit die personenbezogenen Daten nicht bei der betroffenen Person erhoben werden.\footnote{Paal/Pauly DSGVO, \textit{Paal/Hennemann}, Art. 14 Rn. 1.}
Die Ausübung dieser Informationspflicht hat gesetzlich innerhalb eines Monats nach Erlangung der personenbezogenen Daten zu erfolgen.\footnote{Paal/Pauly DSGVO, \textit{Paal/Hennemann}, Art. 14 Rn. 34.}\\
Eine solche Pflicht steht dem Sinn eines Hinweisgebersystems und den Interessen sowohl des Unternehmens als auch des Hinweisgebers entgegen.\footnote{Altenbach/Dierkes, CCZ 2020, 126.}
Einerseits ist es fraglich, wie die Informationspflicht mit dem Interesse des Hinweisgebers anonym zu bleiben vereinbar ist, und andererseits könnte eine vorzeitige Unterrichtung eine Warnfunktion haben und zu einer Verdunkelungsgefahr führen, weil es dem Beschuldigten zum Beispiel ermöglicht wird Beweismittel zu verändern oder auf Zeugen einzuwirken.\footnote{Altenbach/Dierkes, CCZ 2020, 126.}
Abhilfe wird hier durch Art. 14 V b) DSGVO geschaffen.
Dort wird geregelt, dass die Unterrichtung so lange hinausgezögert werden darf, wie das erhebliche Risiko besteht, dass infolge der fristgerechten Umsetzung die Untersuchung der gemeldeten Vorwürfe oder die Erhebung der erforderlichen Beweise gefährdet wird.\footnote{Altenbach/Dierkes, CCZ 2020, 126.}
Weiter kann die Anonymität des Hinweisgebers durch Art. 16 WBRL gesichert werden, welcher das Vertraulichkeitsgebot regelt. 
Der Schutz hieraus ist jedoch nicht absolut.\footnote{Dilling, CCZ 2019, 214.}
Zwar haben nach Art. 16 I WBRL die Mitgliedstaaten sicherzustellen, dass die Identität des Hinweisgebers und auch alle Informationen, aus denen sich die Identität ableiten ließe, ohne dessen ausdrücklicher Zustimmung keinen anderen Personen außer den befugten Mitarbeitern gegenüber offengelegt wird, jedoch darf sie nach Art. 16 II WBRL dann offengelegt werden, wenn dies nach Unionsrecht oder nationalem Recht eine notwendige und verhältnismäßige Pflicht im Rahmen von Untersuchungen durch Behörden darstellt.\footnote{Dilling, CCZ 2019, 214.}
Eine Offenlegung der Identität ist jedoch in der Regel vorerst nicht zu befürchten.\footnote{Altenbach/Dierkes, CCZ 2020, 126.}\\
Im Zusammenhang mit den Meldungen in Hinweisgebersystemen bildet  ferner die Löschungsverpflichtung aus Art. 17 I a) DSGVO eine Herausforderung.\footnote{Altenbach/Dierkes, CCZ 2020, 126.}
Nach Art. 17 I a) DSGVO müssen die personenbezogenen Daten gelöscht werden, wenn diese für die Zwecke, für die sie erhoben oder auf sonstige Weise verarbeitet worden sind, nicht mehr notwendig sind.\footnote{Paal/Pauly DSGVO, \textit{Paal}, Art. 17 Rn. 23.}\\
Ein mögliches Problem, welches sich daraus ergeben kann, wäre, dass zum Beispiel ein Unternehmen, dass pflichtgemäß einem Hinweis eines Mitarbeiters nachgeht, aber keine entsprechenden Verstöße feststellen konnte und schließlich der Löschungsverpflichtung nachkommt. 
Nun könnte dieser Mitarbeiter anschließend mit seinem Hinweis an die Öffentlichkeit gehen, wodurch dem Unternehmen medienwirksam unterstellt werden würde, dass es auf ernste Hinweise seiner Arbeitnehmer nicht reagieren würde.\footnote{Altenbach/Dierkes, CCZ 2020, 126.}
Eben dies würde zu einem erheblichen Imageschaden führen, und obwohl dem Unternehmen nichts vorzuwerfen ist, kann es sich nicht entlasten, da es durch die Löschung der Fallakte mit sämtlichen Dokumenten, Protokollen, etc. nichts vorweisen kann.\footnote{Altenbach/Dierkes, CCZ 2020, 126.}\\
Ein weiters Problem in Verbindung mit Art. 17 DSGVO ergibt sich aus dem allumfassenden Verbot von Repressalien aus Art. 19 WBRL und der Beweislastumkehr aus Art. 21 V WBRL.
Die Löschungsverpflichtung hat für Arbeitnehmer das Potenzial in Arbeitsgerichtsprozessen zu einem erfolgversprechenden Verteidigungsmittel zu werden.\footnote{Altenbach/Dierkes, CCZ 2020, 126.}
Kann der Arbeitnehmer beweisen, dass zuvor eine Meldung als Hinweisgeber getätigt zu haben, und behauptet er anschließend, nach der Löschung der Fallakte, dass die arbeitsgeberseitige Benachteiligung in Folge dieser Meldung geschehen ist, muss das Unternehmen nach Art. 21 V WBRL nachweisen, dass keine Repressalie i.S.d. Art. 19 WBRL vorliegt.\footnote{Altenbach/Dierkes, CCZ 2020, 126.}
Der Arbeitnehmer könnte sich mithin einen eigenen faktischen Kündigungsschutz herbeiführen.\footnote{Thüsing/Rombey, NZG 2018, 1001}
Kann das Unternehmen keinen Gegenbeweis vorlegen wird es zudem doppelt benachteiligt: Zum einen wird die benachteiligende Maßnahme gegen den Arbeitnehmer als unwirksam erklärt, und zum anderen ordnet Art. 23 I b), c) WBRL an, dass gegen Personen und Unternehmen, die Hinweisgebern Benachteiligungen auferlegen oder gegen diese mutwillig Gerichtsverfahren anstrengen, wirksame, angemessene und abschreckende Sanktionen verhängt werden sollen.\footnote{Altenbach/Dierkes, CCZ 2020, 126.}\\
Eine Lösung dieser Problematik ergibt sich zum Teil aus Art. 32 I a), ErwG 28, 29, 75, 78 DSGVO: die Fallakte kann zulässigerweise, statt ganz gelöscht, anonymisiert und pseudonymisiert werden.\footnote{Altenbach/Dierkes, CCZ 2020, 126.}
Damit diese dann dennoch Beweiskraft entfalten können, sollte der Sachverhalt, sowie die konkret ergriffenen Schritte so genau wie möglich dargestellt werden.\footnote{Altenbach/Dierkes, CCZ 2020, 126.}
Es sollte aber dennoch eine Abwägung stattfinden und nicht jede Akte vorbehaltlos gelagert werden: zu bedenken ist, dass sollten tatsächlich Rechtsverstöße des Unternehmens dort dokumentiert sein,  diese im Falle einer Beschlagnahmung durch Aufsichts- und Ermittlungsbehörden weiter verfolgt werden könnten.\footnote{Altenbach/Dierkes, CCZ 2020, 126.}
Es ist mithin zu empfehlen, dass die Akten nur so lange aufbewahrt werden, wie dies zur Wahrung der berechtigten Interessen des Unternehmens erforderlich ist.



\subsubsection{Strafrechtliche Herausforderungen}
Auch im Zusammenhang mit dem Strafrecht haben sich mit der neuen Richtlinie Herausforderungen ergeben.
Hinweisgeber könnten sich wegen der Offenlegung von Geschäftsgeheimnissen strafbar machen.
Diese Strafbarkeit ergabt sich früher aus § 17 I UWG.\footnote{Brockhaus, ZIS 3/2020, 103.}
Am 08.06.2016 wurde dann von der EU die Richtlinie zum Schutz von Geschäftsgeheimnissen (Richtlinie (EU) 2016/943) erlassen, welche dann am 18.04.2019 in Form des Gesetz zum Schutz von Geschäftsgeheimnissen (GeschGehG) in nationales Recht umgesetzt wurde.
Die Strafbarkeit von Hinweisgebern wird nun anhand des GeschGehG festgestellt.\\
Bei der Offenbarung von nicht offenkundigen Tatsachen, die sich auf rechtswidrige Praktiken eines Unternehmens (sog. „illegale Geheimnisse“) beziehen wird, bei der alten Rechtslage nach § 17 UWG, nach einer Ansicht vertreten, dass es mangels eines „berechtigten wirtschaftlichen Interesses“ bereits das Tatobjekt Geschäftsgeheimnis zu verneinen ist, sodass der Hinweisgeber es straflos weitergeben könnte (sog. Tatbestandslösung).\footnote{Reinbacher, KriPoZ 3/2019, 148.}
Nach der anderen Ansicht ist ein Geschäftsgeheimnis zwar anzunehmen, dann aber auf der Ebene der Rechtswidrigkeit eine Abwägung nach § 34 StGB vorzunehmen (Rechtswidrigkeitslösung).\footnote{Reinbacher, KriPoZ 3/2019, 148.}
Die §§ 17-19 UWG wurden nun aber aufgehoben und in § 23 GeschGehG überführt.\footnote{Brockhaus, ZIS 3/2020, 108.}
Eine Strafbarkeit für Whistleblower ergibt sich nun insbesondere aus § 23 I Nr. 3 i.V.m. § 4 II Nr. 3 GeschGehG, wonach sich derjenige strafbar macht, der zur Förderung des eigenen oder fremden Wettbewerbs, aus Eigennutz, zugunsten Dritten oder in der Absicht den Inhaber eines Unternehmens Schaden zuzufügen, als eine bei einem Unternehmen beschäftigte Person ein Geschäftsgeheimnis, das ihr im Rahmen des Beschäftigungsverhältnisses anvertraut worden oder zugänglich geworden ist, während der Geltungsdauer des Beschäftigungsverhältnisses offenlegt.\footnote{Reinbacher, KriPoZ 3/2019, 149.}
Fraglich ist nun, ob hierunter auch die illegalen Geheimnisse fallen sollen.
Hierfür muss zunächst geklärt werden, was unter einem Geschäftsgeheimnis zu verstehen ist.
Dieses ist in § 2 GeschGehG legal definiert.
Unter den Begriff des Geschäftsgeheimnisses fallen Informationen, die a) die weder insgesamt noch in der genauen Anordnung und Zusammensetzung ihrer Bestandteile den Personen in den Kreisen, die üblicherweise mit dieser Art von Informationen umgehen, allgemein bekannt oder ohne Weiteres zugänglich ist und daher von wirtschaftlichem Wert ist und b) die Gegenstand von den Umständen nach angemessenen Geheimhaltungsmaßnahmen durch ihren rechtmäßigen Inhaber ist und c) bei der ein berechtigtes Interesse an der Geheimhaltung besteht.\footnote{§ 2 Nr. 1 GeschGehG.}\\
Nach der ersten Tatbestandslösung scheitert eine Anwendung der §§ 2, 23 GeschGehG auf illegale Geheimnisse an dem Erfordernis des berechtigten Interesses an der Geheimhaltung.\footnote{Reinbacher, KriPoZ 3/2019, 150.}
Es ist aber zu beachten, dass Geschäftsdaten eines Unternehmens gegenüber Dritten nicht weniger schutzbedürftig sind, nur weil sich daraus z.B. eine Steuerhinterziehung ergibt; auch sollen vertrauliche Unterlagen über die Produktion eines Unternehmens wie z.B. Produktionspläne nicht der allgemeinen Kenntnis zustehen, nur weil sich daraus Rechtsverstöße ergeben.\footnote{Reinbacher, KriPoZ 3/2019, 152.}
Es würde sich hieraus für die betroffenen Unternehmen ein ungerechtfertigter Wettbewerbsnachteil ergeben.
Aus diesem Grund muss anerkannt werden, dass auch illegale Geheimnisse vom Geheimnisschutz umfasst.\footnote{Reinbacher, KriPoZ 3/2019, 152.}\\
Nach der zweiten Tatbestandslösung findet eine Rechtfertigung über § 5 Nr. 2 GeschGehG statt.\footnote{Reinbacher, KriPoZ 3/2019, 154.}
In § 5 Nr. 2 GeschGehG heißt es, dass die Offenlegung nicht unter die Verbote des § 4 GeschGehG fällt, wenn dies zum Schutz eines berechtigten Interesses erfolgt, insbesondere zur Aufdeckung einer rechtswidrigen Handlung oder eines beruflichen oder sonstigen Fehlverhaltens, wenn die Erlangung, Nutzung oder Offenlegung geeignet ist, das allgemeine öffentliche Interesse zu schützen.\footnote{§ 4 Nr. 2 GeschGehG.}
Somit sind bei Vorliegen des Tatbestandes sowohl zivilrechtliche Ansprüche aus §§ 6 ff GeschGehG, als auch erst recht die in § 23 GeschGehG vorgesehenen Straftaten ausgeschlossen.\footnote{Brockhaus, ZIS 3/2020, 111, 112.}
Aufgrund eines Übersetzungsfehlers war zunächst das Merkmal der Absicht erforderlich, dieses wurde aber wieder gestrichen.\footnote{Reinbacher, KriPoZ 3/2019, 155.}
Fraglich ist, inwieweit dennoch eine subjektive Komponente zu fordern ist.\footnote{Reinbacher, KriPoZ 3/2019, 156.}
Die Formulierungen „\textit{zum}“ und „\textit{zur}“ deuten auf einen subjektiven Einschlag hin\footnote{Reinbacher, KriPoZ 3/2019, 156.}, es handelt sich bei den Ausnahmetatbeständen um negative Tatbestandsmerkmale.\footnote{Brockhaus, ZIS 3/2020, 114.}
Soll der Hinweisgeber demnach bestraft werden, muss nachgewiesen werden, dass sich sein Vorsatz auf das Nichtvorliegen der Tatbestandsmerkmale erstreckt hat.\footnote{Brockhaus, ZIS 3/2020, 114.}
So muss demnach nachgewiesen werden, dass der Hinweisgeber zum Beispiel gerade nicht zur Aufdeckung einer Straftat gehandelt hat.
Es ist  jedoch nicht erforderlich, dass der Whistleblower in der Absicht handelt das allgemeine öffentliche Interesse zu schützen, es ist lediglich entscheidend, ob durch die Offenlegung das öffentliche Interesse geschützt werde.\footnote{Brockhaus, ZIS 3/2020, 114.} \\
Geht ein Hinweisgeber gutgläubig davon aus, zur Aufdeckung einer Straftat oder eines sonstigen Fehlverhaltens zu handeln, weil er entweder über Tatsachen irrt oder sie falsch bewertet, so ist sein Vorsatz eine Tat nach § 23 GeschGehG zu begehen ausgeschlossen.\footnote{Reinbacher, KriPoZ 3/2019, 157.}
Andererseits würde, wenn ein Ausnahmetatbestand zwar erfüllt ist, der Täter aber hiervon nichts wusste, eine Strafbarkeit wegen eines untauglichen Versuchs in Betracht kommen, vgl. § 23 V GeschGehG.\footnote{Brockhaus, ZIS 3/2020, 115.}
Im Endeffekt ist es mithin für den gutgläubigen Whistleblower nicht entscheidend, ob eine subjektive Komponente nun erforderlich ist oder nicht: denn entweder genügt das subjektive Handeln „zur Aufdeckung einer rechtswidrigen Handlung“ an sich, oder es mangelt am Tatbestandsvorsatz.\footnote{Reinbacher, KriPoZ 3/2019, 157.}\\
Im Rahmen des § 17 UWG hatte nach der herrschenden Meinung in Bezug auf das Whistleblowing eine Interessenabwägung und Verhältnismäßigkeitsprüfung des Einzelfalls stattzufinden.\footnote{Reinbacher, KriPoZ 3/2019, 157.}
Fraglich ist, ob dies mit dem neuen Gesetz immer noch gelten soll.
§ 5 Nr. 2 GeschGehG verlangt nun jedoch lediglich nach einer Geeignetheit, nicht aber nach einer Erforderlichkeit, eine Verhältnismäßigkeitsprüfung ist mithin nach dem Wortlaut nicht erforderlich.\footnote{Ohly, GRUR 2019, 441.}
Auch gesetzessystematisch liegt ein „berechtigtes Interesse“ insbesondere dann vor, wenn der Hinweisgeber „zur Aufdeckung einer rechtswidrigen Handlung“ tätig wird.\footnote{Reinbacher, KriPoZ 3/2019, 158.}
Ferner besteht zum einen die Möglichkeit, dass durch eine Verhältnismäßigkeitsprüfung die in § 5 GeschGehG vorgenommenen gesetzgeberischen Wertungen verwässert werden und so die Entscheidungsvorgänge für Betroffene kaum vorhersehbar werden und zum anderen würde die Komplexität der Normstruktur, die sowieso mit mehreren unbestimmten Rechtsbegriffen operiert, aufgebläht.\footnote{Brockhaus, ZIS 3/2020, 117.}
Eine Verhältnismäßigkeitsprüfung sollt mithin nur in Ausnahmefällen vorgenommen werden.\footnote{Brockhaus, ZIS 3/2020, 117.}\\
Handeln Hinweisgeber im guten Glauben daran, mit ihren Meldungen Missstände in Unternehmen aufzudecken, haben sie mithin nach dem GeschGehG keine strafrechtlichen Konsequenzen wegen des Verrats von Geschäftsgeheimnissen zu befürchten.
\subsection{Eignung des Whistleblowings zur Verhinderung von Straftaten innerhalb eines Unternehmens}

Zuletzt ist fraglich, ob durch das Whistleblowing überhaupt Straftaten innerhalb von Unternehmen verhindert werden können.
Durch Skandale wie beispielsweise der Finanzskandal Luxemburg-Leaks oder die Panama Papers wird deutlich, dass Hinweisgeber eine wichtige Rolle bei der Aufdeckung von Verstößen gegen das EU-Recht spielen können.\footnote{Europäisches Parlament, Whistleblower: Neue Vorschriften für EU-weiten Schutz von Informationen.}
Durch einen umfassenden Hinweisgeberschutz ließen sich in der EU jährlich im Bereich des öffentlichen Auftragswesens geschätzt 5,8 bis 9,6 Mrd. € an Ertragsausfällen einsparen.\footnote{Studie der Europäischen Kommission 2017, 15.}
Whistleblower können durch die Offenlegung von Informationen, die möglicherweise nicht ohne weiteres verfügbar sind rechtswidrige Verfahren aufdecken.
Hierfür sind jedoch klare und umfassende Bestimmungen zum Schutz von Hinweisgebern erforderlich, um Einzelpersonen zu ermutigen Betrug und andere Arten von Fehlverhalten zu melden.\footnote{Studie der Europäischen Kommission 2017, 26.}
Denn Hinweisgeber schrecken aus Angst vor Repressalien häufig davor zurück Missstände zu melden.\footnote{Degenhart/Dziuba, BB 2021/9, 570.}
Ein Schutz der Hinweisgeber vor Repressalien fördert mithin die Aufklärung von Rechtsverstößen oder Missständen, bzw. vermeidet diese von Anfang an.\footnote{Degenhart/Dziuba, BB 2021/9, 571.}
Aus dem Grund, dass Arbeitnehmer eine geringere Hemmschwelle zum Melden von Missständen durch die neue Regelung haben, werden Unternehmen zunächst einmal eher auf diese aufmerksam. 
Zudem kommt die Verpflichtung aus Art. 9 WBRL, wonach sich die Unternehmen auch tatsächlich um die Meldung kümmern müssen. 
Der Hinweis muss mithin auf seinen Wahrheitsgehalt überprüft werden und, sollte er stimmen, der Rechtsverstoß beseitigt werden.
Das wird dazu führen, dass Unternehmen gezwungen sich rechtskonform zu verhalten, sollten sie verhindern wollen, dass ihr Verhalten an die Öffentlichkeit gelangt, was zu erheblichen Einbußen und Repressalien führen könnte.\\
In Anbetracht dessen, könnten Unternehmen sogar eine höhere Hemmschwelle haben gegen Recht zu verstoßen, aus Angst, dass diese Missstände gegebenenfalls an die Öffentlichkeit gelangen.
Es kann also gesagt werden, dass durch den umfassenden Schutz von Hinweisgebern Straftaten innerhalb eines Unternehmens verhindert werden können, da Hinweisgeber nun eher offen dafür sind, etwas gegen diese Missstände zu sagen, wodurch Unternehmen gezwungen sind entsprechende Maßnahmen einzuleiten.

    \section{Fazit}
\label{sec:Fazit}
Die Richtlinie (EU) 019/1937 des Europäischen Parlaments und des Rates zum Schutz von Personen, die Verstöße gegen das Unionsrecht melden, leistet einen wichtigen Beitrag Verstöße gegen geltendes Recht aufzuklären.
Durch sie werden die sogenannten Whistleblower geschützt, indem Unternehmen ein Verbot von jeglichen Repressalien gegenüber Hinweisgebern auferlegt wird.
Verstärkt wird dieser Schutz dadurch, dass das Unternehmen und nicht der Arbeitnehmer nachweisen muss, dass eine arbeitsrechtliche Maßnahme nicht infolge einer getätigten Meldung erfolgt ist.
Der Schutzbereich der Richtlinie ist ebenfalls sehr weit.
Geschützt wird jede Person, die irgendwie im Zusammenhang mit dem Unternehmen steht, jedoch muss diese Person bei Abgabe des Hinweises im guten Glauben gewesen sein, dass ihr Hinweis zutrifft und das Gemeldete einen Verstoß gegen geltendes Recht darstellt.\\
Mit der Implementierung gingen jedoch auch einige Herausforderungen einher.
Für Arbeitgeber stellt die Pflicht zur Errichtung von internen Meldekanälen ein erheblicher Aufwand und Kosten dar.
Zudem treffen ihn vermehrt Dokumentationspflichten, um gegebenenfalls nachweisen zu können, dass eine Maßnahme unabhängig von einem vorher gegebenen Hinweise getroffen wurde.
Hiermit einhergehen Herausforderungen bezüglich der Datenschutzgrundverordnung.
Personenbezogene Daten dürfen nur für einen bestimmten Zeitraum gespeichert werden.
Dies kann durch eine Anonymisierung der Akte umgangen werden.
Auch stellt das Offenlegen von Geschäftsgeheimnissen eine Straftat nach dem GeschGehG dar.
Whistleblower haben jedoch keine Strafverfolgung zu befürchten, sofern sie unter anderem mit ihrem Hinweis rechtswidrige Handlungen aufdecken.\\
In dem Whistleblower stärker geschützt werden und Unternehmen, die diese benachteiligen sanktioniert werden, können Whistleblower vermehrt und ohne Konsequenzen Unternehmen und auch Behörden zur Verantwortung ziehen.
So wird die Transparenz und Integrität in der EU gefördert.
Auch der Rechenschaftspflicht und dem Vertrauen in europäische Organisationen ist die Richtlinie dienlich.
Es ergeben sich zwar einige Herausforderungen mit der neuen Richtlinie, diese lassen sich jedoch alle wie beschrieben bewältigen und stellen keine ernsthafte Gefahr für die Richtlinie und ihrer Wirksamkeit dar.\\
Insgesamt kann gesagt werden, dass die Whistleblowing-Richtlinie eine große Chance für mehr Gerechtigkeit in der Europäischen Union darstellt.

    %==================================================
    %Quellen
    %==================================================

    % \defbibheading{bibnumbered}[Quellenverzeichnis]{
    %     \section{#1}
    % }
    % % \setlength\emergencystretch{5em}
    % \printbibliography[heading=bibnumbered]

    % \newpage


    %==================================================
    %Anhangsverzeichnis 
    %==================================================
    % \addtocontents{toc}{\protect\setcounter{tocdepth}{0}}

    % %Seitennummerierung mit römischen Zahlen
    % \pagenumbering{Roman}
    % %letzte römische Seite wird gespeichert und hier weitergeführt
    % \setcounter{page}{\numexpr\value{romanPage}+1}

    % \appendix
    % % Kapitelnummerierung fortsetzen IV, ...
    % \setcounter{section}{\theromanSection}
    % \renewcommand{\thesection}{\Roman{section}}

    % \parindent0mm

    % \section{Anhangsverzeichnis}

    % \newcommand{\listappendicesname}{Anhangsverzeichnis}
    % \renewcommand{\listappendicesname}{}
    % \newlistof{appendices}{apc}{\listappendicesname}


    % \newcommand{\appendices}[1]{\addcontentsline{apc}{appendices}{#1}}

    % % \GetTitleStringSetup{expand}
    % \newcounter{appendix}

    % \titleformat{\subparagraph}
    % {\normalfont\normalsize\bfseries}{Anhang \protect\the\numexpr\value{appendix}+1\relax{ - }}{0em}{}

    % % \newcommand{\newappendix}[1]{
    % %   \phantomsection
    % %   % \stepcounter{appendix}
    % %   \refstepcounter{appendix}
    % %   \appendices{Anhang \arabic{appendix} - #1}
    % %   \section*{Anhang \arabic{appendix} - #1}
    % %   \label{#1}
    % %   \subparagraph{#1}
    % % }

    % \newcommand{\newappendix}[2]{
    %   \phantomsection
    %   \subparagraph{#1}
    %   \refstepcounter{appendix}
    %   \label{#2}
    %   \appendices{Anhang \arabic{appendix} - #1}
    % }

    % \newcommand{\newappendixpdf}[4]{
    %   \newappendix{#1}{#4}
    %   \foreach \n in {1,...,#3}{
    %     \phantomsection
    %     \label{#4:\n}
    %     \includepdf[pages=\n,pagecommand=\thispagestyle{plain}]{#2}
    %   }
    % }

    % \listofappendices

    % \newpage



    %==================================================
    %Anhang
    %==================================================
    % \section{Anhang}
    % \titleformat*{\section}{\fontsize{12}{14}\selectfont\bfseries} % Schriftgröße und Zeilenabstand ändern
    % \setcounter{page}{1}
    % \renewcommand{\thepage}{A\arabic{page}}

    % \section{Whistleblowing}
\label{sec:Hauptteil}

\subsection{Die EU Whistleblower Richtline}
\label{sec:Teilkapitel}

\subsubsection{Definition Whistleblowing}
Als Whistleblower wird eine Person bezeichnet, die Informationen über Missstände aus dem Innenbereich einer Organisation im öffentlichen Interesse einer Behörde bzw.  der breiteren Öffentlichekit (exterenes Whistleblowing) oder einer organisationsinternen Stelle (internes Whistleblowing) mitteilt.\footnote{Rechtswörterbuch, \textit{Kallos}, Whistleblower.} 
Meist handelt es sich bei einem Whistleblower um einen (etablierten oder ehemaligen) Mitarbeiter oder einen Kunden, welcher aus eigener Erfahrung berichtet.\footnote{Bendel, Whistleblowing.}
Das Whistleblowing ist von der bloßen Beschwerde über persönliche Umstände abzugrenzen, es setzt einen Umstand von allgemeinem Interesse voraus.\footnote{Bendel, Whistleblowing.}


\subsubsection{Aussagen der EU-Whistleblower Richtline}

Die Richtlinie (EU) 2019/1937 des Europäischen Parlaments und des Rates vom 23. Oktober 2019 zum Schutz von Personen, die Verstöße gegen das Unionsrecht melden, auch Whistleblowing-Richtlinie genannt, ist am 23.10.2019 erlassen worden.\footnote{Taschke/Pielow/Volk, NZWiSt 2021, 85.}
Sie wurde am 31.05.2023 durch das Hinweisgeberschutzgesetz (HinSchG) in deutsches Recht umgesetzt.
Durch die Richtlinie werden zwei Hauptziele verfolgt: zum einen sollen Whistleblower eine klar reglementierte Möglichkeit zur Meldung von Missständen haben und zum anderen sollen sie vor Repressialien geschützt werden.\footnote{Dzida/Granetzny, NZA 2020, 1201.}\\
Nun wird genauer auf die Einzelnen Aussagen der Richtlinie eingegangen.

\paragraph{Anwendungsbereich der Richtlinie}
Damit Whistleblower von dem Schutz durch die Richtline profitieren, müssen sie den sachlichen sowie den persönlichen Anwendungsbereich erfüllen.\footnote{Taschke/Pielow/Volk, NZWiSt 2021, 85.}\\
Der sachliche Anwendungsbereich wird in Art. 2 I der Whistleblower-Richtlinie (WBRL) geregelt. 
Der Anwendungsbereich begrenzt sich abschließend auf die dort aufgeführten Verstöße gegen Unionsrecht.\footnote{Taschke/Pielow/Volk, NZWiSt 2021, 85.}
Zum Beispiel sind dies Verstöße in den Bereichen öffentliche Auftragsvergabe, Finanzdienstleistungen, Geldwäsche und Terrorismusfinanzierung, Produktsicherheit, Verkehrssicherheit, Umweltschutz, kerntechnische Sicherheit, Lebensmittel- und Futtermittelsicherheit, Tiergesundheit und Tierschutz, öffentliche Gesundheit, Verbraucherschutz, Schutz der Privatsphäre, Datenschutz, Sicherheit von Netz- und Informationssystemen, EU-Wettbewerbsvorschriften, Körperschaftssteuervorschriften sowie Verstöße gegen die finanziellen Interessen der EU.\footnote{Dzida/Granetzny, NZA 2020, 1201.}
So fällt bloßes unethisches aber nicht rechtswidriges Fehlverhalten nicht in den Anwendungsbereich der Richtlinie.\footnote{Taschke/Pielow/Volk, NZWiSt 2021, 85.}
Ein größerer Anwendungsbereich war aufgrund der beschränkten Gesetzgebungskompetenz der EU nicht möglich, nach Art. 2 II WBLR bleibt es den Mitgliedstaaten jedoch freigestellt den Schutz in Bezug auf Bereiche oder Rechtsakte auszudehnen, die nicht unter Absatz 1 fallen.\footnote{Dzida/Granetzny, NZA 2020, 1201.}
Der deutsche Gesetzgeber ergänzte die Richlinie durch § 2 I Nr. 1, 2 und 10 HinSchG zum einen um „Verstöße, die strafbewehrt sind“, und zum anderen um „Verstöße, die bußgeldbewehrt sind, soweit die verletzte Vorschrift dem Schutz von Leben, Leib oder Gesundheit oder dem Schutz der Rechte von Beschäftigten oder ihrer Vertretungsorgane dient“.\footnote{Dzida/Seibt, NZA 2023, 657.}

\bigbreak

Der persönliche Anwendungsbereich wird in Art. 4 WBRL geregelt.
Aus den Wortlaut ergibt sich, dass der Hinweisgeber „im privaten oder im öffentlichen Sektor tätig“ sein und „im beruflichen Kontext Informationen über Verstöße erlangt haben“ muss.
Es sind demnach im Grunde alle Personen geschützt, die im weitesten Sinne in einer “arbeitsbezogenen Verbinung“ mit dem Unternehmen stehen.\footnote{Dzida/Granetzny, NZA 2020, 1201.}
Das Motiv des Hinweisgebers spielt keine Rolle.\footnote{Schmolke, NZG 2020, 5.}


\paragraph{Voraussetzungen für den Schutz}
In Art. 6 der Whistleblower-Richtlinie wird geregelt, welche Voraussetzungen für den Schutz erfüllt sein müssen.
Hiernach hat ein Hinweisgeber für einen Anspruch auf Schutz, sofern er im Zeitpunkt der Meldung im Hinblick auf die Umstände und die verfügbaren Informationen mit hinreichendem Grund davon ausgehen durfte, dass die gemeldete Information zutrifft und einen Verstoß darstellt, der in den Anwendungsbereich der Richtlinie fällt, vgl. Art. 6 I a)WBRL.\footnote{Taschke/Pielow/Volk, NZWiSt 2021, 85.}
Der Hinweisgeber muss demanch gutgläubig sein, wer also bewusst falsche oder irreführende Informationen meldet ist nicht schutzbedürftig.\footnote{Degenhart/Dziuba, BB 2021/9, 573.}\\
Die Meldung kann zunächst intern, also an eine geeignete Stelle inerhalb des Unternehmens, oder extern an die zuständige Behörde geschehen, vgl. Art 6 I b) WBRL i.V.m. Art. 7, Art. 10 WBRL.\footnote{Schmolke, NZG 2020, 5.}
Die interne und die externe Meldung stehen gleichrangig nebeneinander.\footnote{Dzida/Granetzny, NZA 2020, 1201.}
Höhere Anforderung an den Schutz durch die Richtlinie werden dann gestellt, wenn der Hinweisgeber sich direkt an die Öffentlichekit wendet, vgl. Art. 15 WBRL.\footnote{Schmolke, NZG 2020, 5.}
So muss der Hinweisgeber entweder den Verstoß zunächst intern oder extern gemeldet haben, ohne dass innerhalb der in Art. 11 WBRL festgelegten Höchstfristen von drei bzw. sechs Monaten geeignete Maßnahmen ergriffen worden sind.
Andere Situationen in denen der Hinweisgeber sich direkt an die Öffentlichekit wenden kann sind wenn er davon ausgehen durfte, dass der Verstoß eine „unmittelbare oder offenkundige Gefährdung des öffentlichen Interesses“ darstellt, oder aufgrund der besonderen Umstände des Falls geringe Aussichten bestehen, dass wirksam gegen den Verstoß vorgegangen wird, etwa bei Kollusion zwischen dem Delinquenten und der Behörde (Art. 6 I Buchst. b iVm Art. 15 I WBRL).\footnote{Schmolke, NZG 2020, 5.}
Die Meldung an die Öffentlichekit ist mithin als \textit{ultima ratio} subsidiär\footnote{Dzida/Granetzny, NZA 2020, 1201.} und der Hinweisgeber sollte sich zunächst an die eingerichteten internen oder externen Stellen richten.

\paragraph{Schutzmaßnahmen}
Sind die sachlichen und persönlichen Voraussetzungen erfüllt, wird der Schutz der Hinweisgeber vor allem durch das Verbot jeglicher Repressalien aus Art. 19 ff WBRL gewährleistet.\footnote{Dzida/Granetzny, NZA 2020, 1201.}
Eingeschlossen hierin ist ein Verbot für die Mitgliedstaaten Repressalien nur anzudrohen oder zu versuchen, vgl. Art. 19 WBRL.
Art. 19 WBRL stellt zur besseren Einschätzung einen Katalog an Regelbeispielen für mögliche Repressalien zur Verfügung.
Beispiele hieraus sind unter anderem Suspendierung oder Kündigung, Herabstufung oder Versagung einer Beförderung, Aufgabenverlagerung, negative Leistungsbeurteilung, sowie Nötigung und Diskriminierung.\\
Dieses Verbot wird durch die in Art. 21 V WBRL geregelte Beweislastumkehr verstärkt.
Hiernach muss nun das Unternehmen beweisen, dass es sich bei einer arbeitsrechtlichen Maßnahme nicht um eine unzulässige Repressialie i.S.d. Art. 19 WBRL handelt.\footnote{Dzida/Granetzny, NZA 2020, 1201.}
Somit ist nicht mehr der Arbeitnehmer in der Verantwortung darzulegen, dass eine arbeitsrechtliche Maßnahme aufgrund des Whistleblowings erfolgt ist.
Dies soll potentiellen Beweisschwierigkeiten, die sich für den Hinweisgeber ergeben könnten, entgegen wirken.\footnote{Taschke/Pielow/Volk, NZWiSt 2021, 85.}
Das Verbot jeglicher Repressalien wird ferner durch Art. 23 WBRL untermauert, wonach die Mitgliedstaaten “wirksame, angemessene und abschreckende“ Sanktionen für juristische sowie natürliche Personen festlegen müssen, die Meldungen behindern, Repressalien ergreifen, mutwillig Gerichtsverfahren gegen Hinweisgeber oder geschützte Dritte anstregen oder gegen das Vertraulichkeitsgebot im Rahmen des Meldeverfahrens verstoßen.\footnote{Schmolke, NZG 2020, 5.}
Somit wird sichergestellt, dass Hinweisgeber nach ihrer Meldung keine Konsequenzen zu befürchten haben.

\paragraph{Meldewege}
Potentiellen Hinweisgeber sollen nicht bloß geschützt werden, ihnen soll auch eine effektive Meldeinfrastruktur zur Verfügung gestellt werden.\footnote{Schmolke, NZG 2020, 5.}
Es wird zwischen internen und externen Meldekanälen unterschieden.\footnote{Schmolke, NZG 2020, 5.}
Art. 8 III WBRL gibt vor, dass Unternehmen mit 50 oder mehr Arbeitnehmern einen internen Meldekanal nach Art. 8 I errichten müssen.\footnote{Taschke/Pielow/Volk, NZWiSt 2021, 85.}
Diese müssen nach Art. 9 I a) WBRL zur Gewährleistung der Vertraulichkeit der Identität des Hinweisgebers und erwähnter Personen entsprechend sicher konzipirt, implementiert und betrieben werden.\footnote{Taschke/Pielow/Volk, NZWiSt 2021, 85.}
Zudem muss das Unternehmen dem Hinweisgeber nach Art. 9 I b) WBRL innerhalb von sieben Tagen den Eingang der Meldung bestätigen und ihm nach spätestens drei Monaten berichten, wie mit seinem Hinweis umgegangen wurde und was für Maßnahmen ergriffen worden sind.\footnote{Dzida/Granetzny, NZA 2020, 1201.}
Es wird somit sichergesetellt, dass den Meldungen der Hinweisgeber auch nachgegangen wird.\\
Für die externen Meldekanäle müssen gemäß Art. 10 WBRL durch die Mitgliedstaaten zuständige Behörden benannt werden, die befugt sind Meldungen entgegenzunehmen, Rückmeldung zu geben und entsprechende Folgemaßnahmen zu ergreifen.\footnote{Dzida/Granetzny, NZA 2020, 1201.}
Zu diesem Zweck müssen die Behörden mit angemessenen Ressourcen ausgestattet werden, vgl. Art. 11 I WBRL.
Zur Wahrung des Vertraulichkeitsgebots müssen die Meldekanäle grundsätzlich so konzipirt sein, dass Unbefugte keinen Zugriff auf die Daten erhalten können, vgl. Art 16 WBRL.\footnote{Dzida/Granetzny, NZA 2020, 1201.}



\subsubsection{Beeinflussung der Verabschiedung durch Entwicklungen in den USA}

Das Whistleblowing-Recht in den USA hat dort eine langjährige Tradition.\footnote{Gerdemann, NZA-Beilage 2020, 43.}
Hintergrund ist das Bedürfnis des dortigen Bundesgesetzgebers, trotz der vergleichsweise beschränkten eigenen Administrativressourcen, Verstöße gegen bundesrechtliche Normen in Erfahrung zu bringen und hierdurch eine wirksame praktische Durchsetzung seiner seit dem letzten Jahrhundert sukzessive expandierenden Legislativkompetenzen zu erwirken.\footnote{Gerdemann, NZA-Beilage 2020, 43.}
Dieser Regulierungsmechanismus der USA führte zunehmend zu Erfolgen, sodass sich die Europäische Union das Prinzip des Whistleblowings zu eigen gemacht hat.\footnote{Gerdemann, NZA-Beilage 2020, 43.}
Die USA fungierte mithin als legislatives Vorbild.
Vor allem zwei Akte zogen die Aufmerksamkeit der Europäischen Unions auf sich: Der Sarbanes-Oxley Act von 2002 (SOX) und der Dodd-Frank Act von 2010.\footnote{Gerdemann, SR 1, 2021}\\
Der erst genannte ist ein vom amerikanischen Kongress verabschiedetes Bundesgesetz mit dem Ziel, das Vertrauen der Anleger in die Richtigkeit und Verlässlichkeit der veröffentlichten Finanzdaten von Unternehmen wiederherzuestellen.\footnote{CC, \textit{Obermayr}, § 44 Rn. 111.}
Er wurde als Reaktion auf die Bilanzskandale erlassen.\footnote{Schürrle/Fleck,CCZ 2011, 218.}
Unter anderem wird den betroffenen Unternehmen die Pflicht zur Errichtung eines internen Kontrolsystems sowie die Etablierung eines Beschwerdemanagements (Whistleblower-System) auferlegt.\footnote{Handbuch VertriebsR, \textit{Passarge}, § 79 Rn. 120.}
Von dem Act betroffen sind alle Unternehmen, deren Aktien an der New York Stock Exchange gehandelt werden; hierunter sind auch 23 deutsche Unternehmen.\footnote{CC, \textit{Obermayr}, § 44 Rn. 111.}
Mithin ist der SOX grundsätzlich unabhängig vom Sitz des Unternehmens anwendbar\footnote{Handbuch VertriebsR, \textit{Passarge}, § 79 Rn. 119.}, hierdurch ergibt sich bereits eine Relevanz für die EU.
Bereits im SOX wurden Hinweisgeber vor Diskriminierung in Form von Entlassung, Zurückstufung oder ähnlichem geschützt, bei Verstoß konnten den Arbeitnehmern Schadenersatzansprüche zustehen.\footnote{Schürrle/Fleck,CCZ 2011, 218.}\\
Der Dodd-Frank Act wurde schließlich als Reaktion auf den Finanzskandal 2008 erlassen.\footnote{Schürrle/Fleck,CCZ 2011, 218.}
Durch diesen Act sollte das Antidiskriminierungsrecht des SOX noch verbessert werden und zudem wurde mit dem „SEC Office of the Whistleblower“ eine externe Meldebehörde etabliert.\footnote{Gerdemann, SR 1, 2021}
Durch den Dodd-Frank Act wurde also der Schutz der Hinweisgeber vor Vergeltung ausgeweitet, hierfür wurden Durchführungsbestimmungen vorgegeben, zudem wurde der Anwendungsbereich auf nicht-börsengelistete Tochterunternehmen ausgeweitet.\footnote{Schürrle/Fleck,CCZ 2011, 218.}
Die Europäische Union sah sich mit ähnlichne Herausforderungen wie das Federal Government der USA konfrontiert.\footnote{Gerdemann, SR 1, 2021}
So ist es nicht verwunderlich, dass die Whistleblowing-Richtlinie der Europäischen Union den Regelungen aus dem USA zum Teil ähnelt.\\
Zudem kommt der Faktor, dass durch internationale Whistleblowing-Fälle das öffentliche Interesse am Phänomen des Whistleblowings gestiegen ist.\footnote{Gerdemann, SR 1, 2021} 
Beispiele für bekannte Fälle wären unter anderem Edward Snowden, Luxemburg-Leaks oder die Panama Papers.
Durch diese wurde verstärkt deutlich, dass Hinweisgeber eine wichtige Rolle bei der Aufedeckung von Verstößen gegen das EU-Recht spielen können, die das öffentliche Interesse und das Wohlergehen der Bürger und der Gesellschaft schädigen.\footnote{Yakimova, Whistleblower: Neue Vorschriften für EU-weiten Schutz von Informationen.}
Trotz der großen unionsrechtlichen Relevanz hatten vor Erlass der Richtlinie nur 10 EU-Länder einen umfassenden Schutz für Hinweisgeber.\footnote{Yakimova, Whistleblower: Neue Vorschriften für EU-weiten Schutz von Informationen.}
Eine breit angelegte, sektorübergreifende Regulierung des Whistleblowings in Europa schien dennoch zunächst eher unwahrscheinlich, bis sich eine Koalition aus staatlichem Rechtsdurchsetzungsinteresse und zivilgesellschaftlichem Engagement bildete.\footnote{Gerdemann, SR 1, 2021} 

\subsection{Mit dem Whistleblowing verbunden Herausforderungen}
Die Erlassung der Whistleblowing-Richtlinie bewirkt einen wichtigen Beitrag zum Schutz von Hinweisgebern. Die tatsächliche Umsetzung ist aber nicht ohne Herausforderungen möglich.

\subsubsection{Arbeitsrechtliche Herausforderungen}
Zunächst erwachsen im Arbeitsrecht einige Herausforderungen für die Arbeitgeber.\footnote{Degenhart/Dziuba, BB 2021/9, 571.}
Juristische Personen des Privatrechts werden zur Errichtung eines Hinweisgebersystems verpflichtet.
Auch kleine Unternehmen mit 50 oder mehr Mitarbeitern müssen seit dem 17. Dezember 2023 eine interne Meldestelle eingerichtet haben, wobei sie zusammen mit anderen Unternehmen eine “gemeinsame Meldestelle“ betreiben können, vgl. Art. 8 III WBRL/§ 12 HinSchG.\footnote{Rohrlich, Hinweisgeberschutzgesetz.}
Für die regelmäßige Kontrolle und Bearbeitung der eingehenden Meldungen muss dementsprechend eine zuständige Person oder Stelle benannt werden.\footnote{Rohrlich, Hinweisgeberschutzgesetz.}
Hierfür müssen nicht unerhebliche personelle Ressourcen aufgewendet werden.\\
Die Meldekanäle müssen zudem nach Art. 9 WBRL einige Anforderungen erfüllen. 
Unter anderem müssen diese so sicher konzipiert, eingerichtet und betrieben werden, dass die Vertraulichkeit der Identität des Hinweisgebers gewahrt bleibt, weiter muss dem Hinweisgeber der Eingang seiner Meldung innerhalb von sieben Tagen bestätigt werden und es muss eine Rückmeldung innerhalb von drei Monaten erfolgen.\footnote{Degenhart/Dziuba, BB 2021/9, 572.}
Sollte dies nicht gewährleistet sein, drohen den Unternehmen eine Geldbuße von bis zu 20.000 Euro, da eine Verletzung als Ordnungswidrigkeit gewertet wird.\footnote{Rohrlich, Hinweisgeberschutzgesetz.}\\
Ein weiterer Punkt, der arbeitsrechtlich zu Herausforderungen führen könnte, könnte die Beweislastumkehr aus Art. 21 V WBRL sein.
Dieses System kann potenziell von Arbeitnehmern ausgenutzt werden, indem diese sich durch einen rechtzeitigen „Hinweis“ zusätzlichen Kündigungsschutz verschaffen.\footnote{Dzida/Granetzny, NZA 2020, 1201.}
Der Arbeitgeber müsste nun erst einmal nachweisen, dass er den Arbeitnehmer nicht aufgrund des Hinweises kündigen wollte.
Dies ist vor allem bei wie der Probezeitkündigung, in denen eine rechtliche Notwendigkeit zur Begründung einer Maßnahme normalerweise gar nicht besteht, relevant.\footnote{Degenhart/Dziuba, BB 2021/9, 573.}
Der Arbeitgeber muss nun faktisch doch begründen, warum die Kündigung stattfand, um zu beiweisen, dass sie nicht auf dem Hinweis beruhte.\footnote{Degenhart/Dziuba, BB 2021/9, 573.}
Es ist zur Verhinderung eine umfassende Dokumentation der Unternehmen von zum Beispiel Mitarbeiterbewertungen, Karriereentwicklungen, erteilte Abmahnungen sowie bereits aufgetretene Probleme, notwendig, sodass gegebenenfalls im Kündigungsprozess nachgewiesen werden kann, dass dieser nicht mit dem vermeintlichen Whistleblowing zusammenhängt.\footnote{Dzida/Granetzny, NZA 2020, 1201.}
Mit dieser umfassenden Dokumentation ist aber eben auch ein enormer Aufwand verbunden, den die Arbeitgeber nun zusätzlich zu bewältigen haben.



\subsubsection{Datenrechtliche Herausforderungen}
Auch in Verbindung mit der Datenschutzgrundverordnung (DSGVO) kann es zu Spannungen kommen.
Gem. Art. 1 DSGVO zielt diese Verordnung darauf ab natürliche Personen bei der Verarbeitung personenbezogener Daten zu schützen.
Ferner sollen die Grundrechte und Grundfreiheiten natürlicher Personen und insbesondere deren Recht auf Schutz personenbezogener Daten geschützt werden.
Für die Verarbeitung personenbezogener Daten ist immer eine Rechtsgrundlage erforderlich.\footnote{Altenbach/Dierkes, CCZ 2020, 126.}
Bislang ergab sich diese in Bezug auf Hinweisgebersysteme gem. Art. 6 I 1 c) DSGVO nur für Unternehmen in ausgewählten Wirtschaftsbereichen.
Durch die Umsetzung der Whistleblowing-Richtlinie ergibt sich aus der nationalen Umsetzungsnorm zu Art. 8 WBRL eine entsprechende Rechtsgrundlage.\footnote{Altenbach/Dierkes, CCZ 2020, 126.}
In Deutschland ergibt sich diese aus § 12 HinSchG.
Hiervon können jedoch nur Unternehmen gebrauch machen, welche in den Anwendungsbereich der Whistleblowing-Richtlinie fallen.
Somit müssen vor allem Unternehmen mit weniger als 50 Mitarbeitern weiterhin gemäß Art. 6 I 1 f) DSGVO abwägen, ob die Verarbeitung zur Wahrung der berechtigten Interessen des Unternehmens erforderlich ist und die Interessen oder die Grundrechte der betreffenden Person nicht überwiegt.\footnote{Altenbach/Dierkes, CCZ 2020, 126.}\\
Ferner könnte es zu Konflikten mit Art. 14 DSGVO kommen.
In Art. 14 DSGVO wird die Art und der Umfang der Informationspflicht des Verantwortlichen gegenüber der betroffenen Person geregelt, wenn und soweit die personenbezogenen Daten nicht bei der betroffenen Person erhoben werden.\footnote{Paal/Pauly DSGVO, \textit{Paal/Hennemann}, Art. 14 Rn. 1.}
Die Ausübung dieser Informationspflicht hat gesetzlich innerhalb eines Monats nach Erlangung der personenbezogenen Daten zu erfolgen.\footnote{Paal/Pauly DSGVO, \textit{Paal/Hennemann}, Art. 14 Rn. 34.}\\
Eine solche Pflicht steht dem Sinn eines Hinweisgebersystems und den Interessen sowohl des Unternehmens als auch des Hinweisgebers entgegen.\footnote{Altenbach/Dierkes, CCZ 2020, 126.}
Einerseits ist es fraglich, wie die Informationspflicht mit dem Interesse des Hinweisgebers anonym zu bleiben vereinbar ist, und andererseits könnte eine vorzeitige Unterrichtung eine Warnfunktion haben und zu einer Verdunkelungsgefahr führen, weil es dem Beschuldigten zum Beispiel ermöglicht wird Beweismittel zu verändern oder auf Zeugen einzuwirken.\footnote{Altenbach/Dierkes, CCZ 2020, 126.}
Abhilfe wird hier durch Art. 14 V b) DSGVO geschaffen.
Dort wird geregelt, dass die Unterrichtung so lange hinausgezögert werden darf, wie das erhebliche Risiko besteht, dass infolge der fristgerechten Umsetzung die Untersuchung der gemeldeten Vorwürfe oder die Erhebung der erforderlichen Beweise gefährdet wird.\footnote{Altenbach/Dierkes, CCZ 2020, 126.}
Weiter kann die Anonymität des Hinweisgebers durch Art. 16 WBRL gesichert werden, welcher das Vertraulichkeitsgebot regelt. 
Der Schutz hieraus ist jedoch nicht absolut.\footnote{Dilling, CCZ 2019, 214.}
Zwar haben nach Art. 16 I WBRL die Mitgliedstaaten sicherzustellen, dass die Identität des Hinweisgebers und auch alle Informationen, aus denen sich die Identität ableiten ließe, ohne dessen ausdrücklicher Zustimmung keinen anderen Personen außer den befugten Mitarbeitern gegenüber offengelegt wird, jedoch darf sie nach Art. 16 II WBRL dann offengelegt werden, wenn dies nach Unionsrecht oder nationalem Recht eine notwendige und verhältnismäßige Pflicht im Rahmen von Untersuchungen durch Behörden darstellt.\footnote{Dilling, CCZ 2019, 214.}
Eine Offenlegung der Identität ist jedoch in der Regel vorerst nicht zu befürchten.\footnote{Altenbach/Dierkes, CCZ 2020, 126.}\\
Im Zusammenhang mit den Meldungen in Hinweisgebersystemen bildet  ferner die Löschungsverpflichtung aus Art. 17 I a) DSGVO eine Herausforderung.\footnote{Altenbach/Dierkes, CCZ 2020, 126.}
Nach Art. 17 I a) DSGVO müssen die personenbezogenen Daten gelöscht werden, wenn diese für die Zwecke, für die sie erhoben oder auf sonstige Weise verarbeitet worden sind, nicht mehr notwendig sind.\footnote{Paal/Pauly DSGVO, \textit{Paal}, Art. 17 Rn. 23.}\\
Ein mögliches Problem, welches sich daraus ergeben kann, wäre, dass zum Beispiel ein Unternehmen, dass pflichtgemäß einem Hinweis eines Mitarbeiters nachgeht, aber keine entsprechenden Verstöße feststellen konnte und schließlich der Löschungsverpflichtung nachkommt. 
Nun könnte dieser Mitarbeiter anschließend mit seinem Hinweis an die Öffentlichkeit gehen, wodurch dem Unternehmen medienwirksam unterstellt werden würde, dass es auf ernste Hinweise seiner Arbeitnehmer nicht reagieren würde.\footnote{Altenbach/Dierkes, CCZ 2020, 126.}
Eben dies würde zu einem erheblichen Imageschaden führen, und obwohl dem Unternehmen nichts vorzuwerfen ist, kann es sich nicht entlasten, da es durch die Löschung der Fallakte mit sämtlichen Dokumenten, Protokollen, etc. nichts vorweisen kann.\footnote{Altenbach/Dierkes, CCZ 2020, 126.}\\
Ein weiters Problem in Verbindung mit Art. 17 DSGVO ergibt sich aus dem allumfassenden Verbot von Repressalien aus Art. 19 WBRL und der Beweislastumkehr aus Art. 21 V WBRL.
Die Löschungsverpflichtung hat für Arbeitnehmer das Potenzial in Arbeitsgerichtsprozessen zu einem erfolgversprechenden Verteidigungsmittel zu werden.\footnote{Altenbach/Dierkes, CCZ 2020, 126.}
Kann der Arbeitnehmer beweisen, dass zuvor eine Meldung als Hinweisgeber getätigt zu haben, und behauptet er anschließend, nach der Löschung der Fallakte, dass die arbeitsgeberseitige Benachteiligung in Folge dieser Meldung geschehen ist, muss das Unternehmen nach Art. 21 V WBRL nachweisen, dass keine Repressalie i.S.d. Art. 19 WBRL vorliegt.\footnote{Altenbach/Dierkes, CCZ 2020, 126.}
Der Arbeitnehmer könnte sich mithin einen eigenen faktischen Kündigungsschutz herbeiführen.\footnote{Thüsing/Rombey, NZG 2018, 1001}
Kann das Unternehmen keinen Gegenbeweis vorlegen wird es zudem doppelt benachteiligt: Zum einen wird die benachteiligende Maßnahme gegen den Arbeitnehmer als unwirksam erklärt, und zum anderen ordnet Art. 23 I b), c) WBRL an, dass gegen Personen und Unternehmen, die Hinweisgebern Benachteiligungen auferlegen oder gegen diese mutwillig Gerichtsverfahren anstrengen, wirksame, angemessene und abschreckende Sanktionen verhängt werden sollen.\footnote{Altenbach/Dierkes, CCZ 2020, 126.}\\
Eine Lösung dieser Problematik ergibt sich zum Teil aus Art. 32 I a), ErwG 28, 29, 75, 78 DSGVO: die Fallakte kann zulässigerweise, statt ganz gelöscht, anonymisiert und pseudonymisiert werden.\footnote{Altenbach/Dierkes, CCZ 2020, 126.}
Damit diese dann dennoch Beweiskraft entfalten können, sollte der Sachverhalt, sowie die konkret ergriffenen Schritte so genau wie möglich dargestellt werden.\footnote{Altenbach/Dierkes, CCZ 2020, 126.}
Es sollte aber dennoch eine Abwägung stattfinden und nicht jede Akte vorbehaltlos gelagert werden: zu bedenken ist, dass sollten tatsächlich Rechtsverstöße des Unternehmens dort dokumentiert sein,  diese im Falle einer Beschlagnahmung durch Aufsichts- und Ermittlungsbehörden weiter verfolgt werden könnten.\footnote{Altenbach/Dierkes, CCZ 2020, 126.}
Es ist mithin zu empfehlen, dass die Akten nur so lange aufbewahrt werden, wie dies zur Wahrung der berechtigten Interessen des Unternehmens erforderlich ist.



\subsubsection{Strafrechtliche Herausforderungen}
Auch im Zusammenhang mit dem Strafrecht haben sich mit der neuen Richtlinie Herausforderungen ergeben.
Hinweisgeber könnten sich wegen der Offenlegung von Geschäftsgeheimnissen strafbar machen.
Diese Strafbarkeit ergabt sich früher aus § 17 I UWG.\footnote{Brockhaus, ZIS 3/2020, 103.}
Am 08.06.2016 wurde dann von der EU die Richtlinie zum Schutz von Geschäftsgeheimnissen (Richtlinie (EU) 2016/943) erlassen, welche dann am 18.04.2019 in Form des Gesetz zum Schutz von Geschäftsgeheimnissen (GeschGehG) in nationales Recht umgesetzt wurde.
Die Strafbarkeit von Hinweisgebern wird nun anhand des GeschGehG festgestellt.\\
Bei der Offenbarung von nicht offenkundigen Tatsachen, die sich auf rechtswidrige Praktiken eines Unternehmens (sog. „illegale Geheimnisse“) beziehen wird, bei der alten Rechtslage nach § 17 UWG, nach einer Ansicht vertreten, dass es mangels eines „berechtigten wirtschaftlichen Interesses“ bereits das Tatobjekt Geschäftsgeheimnis zu verneinen ist, sodass der Hinweisgeber es straflos weitergeben könnte (sog. Tatbestandslösung).\footnote{Reinbacher, KriPoZ 3/2019, 148.}
Nach der anderen Ansicht ist ein Geschäftsgeheimnis zwar anzunehmen, dann aber auf der Ebene der Rechtswidrigkeit eine Abwägung nach § 34 StGB vorzunehmen (Rechtswidrigkeitslösung).\footnote{Reinbacher, KriPoZ 3/2019, 148.}
Die §§ 17-19 UWG wurden nun aber aufgehoben und in § 23 GeschGehG überführt.\footnote{Brockhaus, ZIS 3/2020, 108.}
Eine Strafbarkeit für Whistleblower ergibt sich nun insbesondere aus § 23 I Nr. 3 i.V.m. § 4 II Nr. 3 GeschGehG, wonach sich derjenige strafbar macht, der zur Förderung des eigenen oder fremden Wettbewerbs, aus Eigennutz, zugunsten Dritten oder in der Absicht den Inhaber eines Unternehmens Schaden zuzufügen, als eine bei einem Unternehmen beschäftigte Person ein Geschäftsgeheimnis, das ihr im Rahmen des Beschäftigungsverhältnisses anvertraut worden oder zugänglich geworden ist, während der Geltungsdauer des Beschäftigungsverhältnisses offenlegt.\footnote{Reinbacher, KriPoZ 3/2019, 149.}
Fraglich ist nun, ob hierunter auch die illegalen Geheimnisse fallen sollen.
Hierfür muss zunächst geklärt werden, was unter einem Geschäftsgeheimnis zu verstehen ist.
Dieses ist in § 2 GeschGehG legal definiert.
Unter den Begriff des Geschäftsgeheimnisses fallen Informationen, die a) die weder insgesamt noch in der genauen Anordnung und Zusammensetzung ihrer Bestandteile den Personen in den Kreisen, die üblicherweise mit dieser Art von Informationen umgehen, allgemein bekannt oder ohne Weiteres zugänglich ist und daher von wirtschaftlichem Wert ist und b) die Gegenstand von den Umständen nach angemessenen Geheimhaltungsmaßnahmen durch ihren rechtmäßigen Inhaber ist und c) bei der ein berechtigtes Interesse an der Geheimhaltung besteht.\footnote{§ 2 Nr. 1 GeschGehG.}\\
Nach der ersten Tatbestandslösung scheitert eine Anwendung der §§ 2, 23 GeschGehG auf illegale Geheimnisse an dem Erfordernis des berechtigten Interesses an der Geheimhaltung.\footnote{Reinbacher, KriPoZ 3/2019, 150.}
Es ist aber zu beachten, dass Geschäftsdaten eines Unternehmens gegenüber Dritten nicht weniger schutzbedürftig sind, nur weil sich daraus z.B. eine Steuerhinterziehung ergibt; auch sollen vertrauliche Unterlagen über die Produktion eines Unternehmens wie z.B. Produktionspläne nicht der allgemeinen Kenntnis zustehen, nur weil sich daraus Rechtsverstöße ergeben.\footnote{Reinbacher, KriPoZ 3/2019, 152.}
Es würde sich hieraus für die betroffenen Unternehmen ein ungerechtfertigter Wettbewerbsnachteil ergeben.
Aus diesem Grund muss anerkannt werden, dass auch illegale Geheimnisse vom Geheimnisschutz umfasst.\footnote{Reinbacher, KriPoZ 3/2019, 152.}\\
Nach der zweiten Tatbestandslösung findet eine Rechtfertigung über § 5 Nr. 2 GeschGehG statt.\footnote{Reinbacher, KriPoZ 3/2019, 154.}
In § 5 Nr. 2 GeschGehG heißt es, dass die Offenlegung nicht unter die Verbote des § 4 GeschGehG fällt, wenn dies zum Schutz eines berechtigten Interesses erfolgt, insbesondere zur Aufdeckung einer rechtswidrigen Handlung oder eines beruflichen oder sonstigen Fehlverhaltens, wenn die Erlangung, Nutzung oder Offenlegung geeignet ist, das allgemeine öffentliche Interesse zu schützen.\footnote{§ 4 Nr. 2 GeschGehG.}
Somit sind bei Vorliegen des Tatbestandes sowohl zivilrechtliche Ansprüche aus §§ 6 ff GeschGehG, als auch erst recht die in § 23 GeschGehG vorgesehenen Straftaten ausgeschlossen.\footnote{Brockhaus, ZIS 3/2020, 111, 112.}
Aufgrund eines Übersetzungsfehlers war zunächst das Merkmal der Absicht erforderlich, dieses wurde aber wieder gestrichen.\footnote{Reinbacher, KriPoZ 3/2019, 155.}
Fraglich ist, inwieweit dennoch eine subjektive Komponente zu fordern ist.\footnote{Reinbacher, KriPoZ 3/2019, 156.}
Die Formulierungen „\textit{zum}“ und „\textit{zur}“ deuten auf einen subjektiven Einschlag hin\footnote{Reinbacher, KriPoZ 3/2019, 156.}, es handelt sich bei den Ausnahmetatbeständen um negative Tatbestandsmerkmale.\footnote{Brockhaus, ZIS 3/2020, 114.}
Soll der Hinweisgeber demnach bestraft werden, muss nachgewiesen werden, dass sich sein Vorsatz auf das Nichtvorliegen der Tatbestandsmerkmale erstreckt hat.\footnote{Brockhaus, ZIS 3/2020, 114.}
So muss demnach nachgewiesen werden, dass der Hinweisgeber zum Beispiel gerade nicht zur Aufdeckung einer Straftat gehandelt hat.
Es ist  jedoch nicht erforderlich, dass der Whistleblower in der Absicht handelt das allgemeine öffentliche Interesse zu schützen, es ist lediglich entscheidend, ob durch die Offenlegung das öffentliche Interesse geschützt werde.\footnote{Brockhaus, ZIS 3/2020, 114.} \\
Geht ein Hinweisgeber gutgläubig davon aus, zur Aufdeckung einer Straftat oder eines sonstigen Fehlverhaltens zu handeln, weil er entweder über Tatsachen irrt oder sie falsch bewertet, so ist sein Vorsatz eine Tat nach § 23 GeschGehG zu begehen ausgeschlossen.\footnote{Reinbacher, KriPoZ 3/2019, 157.}
Andererseits würde, wenn ein Ausnahmetatbestand zwar erfüllt ist, der Täter aber hiervon nichts wusste, eine Strafbarkeit wegen eines untauglichen Versuchs in Betracht kommen, vgl. § 23 V GeschGehG.\footnote{Brockhaus, ZIS 3/2020, 115.}
Im Endeffekt ist es mithin für den gutgläubigen Whistleblower nicht entscheidend, ob eine subjektive Komponente nun erforderlich ist oder nicht: denn entweder genügt das subjektive Handeln „zur Aufdeckung einer rechtswidrigen Handlung“ an sich, oder es mangelt am Tatbestandsvorsatz.\footnote{Reinbacher, KriPoZ 3/2019, 157.}\\
Im Rahmen des § 17 UWG hatte nach der herrschenden Meinung in Bezug auf das Whistleblowing eine Interessenabwägung und Verhältnismäßigkeitsprüfung des Einzelfalls stattzufinden.\footnote{Reinbacher, KriPoZ 3/2019, 157.}
Fraglich ist, ob dies mit dem neuen Gesetz immer noch gelten soll.
§ 5 Nr. 2 GeschGehG verlangt nun jedoch lediglich nach einer Geeignetheit, nicht aber nach einer Erforderlichkeit, eine Verhältnismäßigkeitsprüfung ist mithin nach dem Wortlaut nicht erforderlich.\footnote{Ohly, GRUR 2019, 441.}
Auch gesetzessystematisch liegt ein „berechtigtes Interesse“ insbesondere dann vor, wenn der Hinweisgeber „zur Aufdeckung einer rechtswidrigen Handlung“ tätig wird.\footnote{Reinbacher, KriPoZ 3/2019, 158.}
Ferner besteht zum einen die Möglichkeit, dass durch eine Verhältnismäßigkeitsprüfung die in § 5 GeschGehG vorgenommenen gesetzgeberischen Wertungen verwässert werden und so die Entscheidungsvorgänge für Betroffene kaum vorhersehbar werden und zum anderen würde die Komplexität der Normstruktur, die sowieso mit mehreren unbestimmten Rechtsbegriffen operiert, aufgebläht.\footnote{Brockhaus, ZIS 3/2020, 117.}
Eine Verhältnismäßigkeitsprüfung sollt mithin nur in Ausnahmefällen vorgenommen werden.\footnote{Brockhaus, ZIS 3/2020, 117.}\\
Handeln Hinweisgeber im guten Glauben daran, mit ihren Meldungen Missstände in Unternehmen aufzudecken, haben sie mithin nach dem GeschGehG keine strafrechtlichen Konsequenzen wegen des Verrats von Geschäftsgeheimnissen zu befürchten.
\subsection{Eignung des Whistleblowings zur Verhinderung von Straftaten innerhalb eines Unternehmens}

Zuletzt ist fraglich, ob durch das Whistleblowing überhaupt Straftaten innerhalb von Unternehmen verhindert werden können.
Durch Skandale wie beispielsweise der Finanzskandal Luxemburg-Leaks oder die Panama Papers wird deutlich, dass Hinweisgeber eine wichtige Rolle bei der Aufdeckung von Verstößen gegen das EU-Recht spielen können.\footnote{Europäisches Parlament, Whistleblower: Neue Vorschriften für EU-weiten Schutz von Informationen.}
Durch einen umfassenden Hinweisgeberschutz ließen sich in der EU jährlich im Bereich des öffentlichen Auftragswesens geschätzt 5,8 bis 9,6 Mrd. € an Ertragsausfällen einsparen.\footnote{Studie der Europäischen Kommission 2017, 15.}
Whistleblower können durch die Offenlegung von Informationen, die möglicherweise nicht ohne weiteres verfügbar sind rechtswidrige Verfahren aufdecken.
Hierfür sind jedoch klare und umfassende Bestimmungen zum Schutz von Hinweisgebern erforderlich, um Einzelpersonen zu ermutigen Betrug und andere Arten von Fehlverhalten zu melden.\footnote{Studie der Europäischen Kommission 2017, 26.}
Denn Hinweisgeber schrecken aus Angst vor Repressalien häufig davor zurück Missstände zu melden.\footnote{Degenhart/Dziuba, BB 2021/9, 570.}
Ein Schutz der Hinweisgeber vor Repressalien fördert mithin die Aufklärung von Rechtsverstößen oder Missständen, bzw. vermeidet diese von Anfang an.\footnote{Degenhart/Dziuba, BB 2021/9, 571.}
Aus dem Grund, dass Arbeitnehmer eine geringere Hemmschwelle zum Melden von Missständen durch die neue Regelung haben, werden Unternehmen zunächst einmal eher auf diese aufmerksam. 
Zudem kommt die Verpflichtung aus Art. 9 WBRL, wonach sich die Unternehmen auch tatsächlich um die Meldung kümmern müssen. 
Der Hinweis muss mithin auf seinen Wahrheitsgehalt überprüft werden und, sollte er stimmen, der Rechtsverstoß beseitigt werden.
Das wird dazu führen, dass Unternehmen gezwungen sich rechtskonform zu verhalten, sollten sie verhindern wollen, dass ihr Verhalten an die Öffentlichkeit gelangt, was zu erheblichen Einbußen und Repressalien führen könnte.\\
In Anbetracht dessen, könnten Unternehmen sogar eine höhere Hemmschwelle haben gegen Recht zu verstoßen, aus Angst, dass diese Missstände gegebenenfalls an die Öffentlichkeit gelangen.
Es kann also gesagt werden, dass durch den umfassenden Schutz von Hinweisgebern Straftaten innerhalb eines Unternehmens verhindert werden können, da Hinweisgeber nun eher offen dafür sind, etwas gegen diese Missstände zu sagen, wodurch Unternehmen gezwungen sind entsprechende Maßnahmen einzuleiten.
    

\end{document}