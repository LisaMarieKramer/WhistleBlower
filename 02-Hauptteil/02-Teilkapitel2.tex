\subsection{Mit dem Whistleblowing verbunden Herausforderungen}
Die Erlassung der Whistleblowing-Richtlinie bewirkt einen wichtigen Beitrag zum Schutz von Hinweisgebern. Die tatsächliche Umsetzung ist aber nicht ohne Herausforderungen möglich.

\subsubsection{Arbeitsrechtliche Herausforderungen}
Zunächst erwachsen im Arbeitsrecht einige Herausforderungen für die Arbeitgeber.\footnote{Degenhart/Dziuba, BB 2021/9, 571.}
Juristische Personen des Privatrechts werden zur Errichtung eines Hinweisgebersystems verpflichtet.
Auch kleine Unternehmen mit 50 oder mehr Mitarbeitern müssen seit dem 17. Dezember 2023 eine interne Meldestelle eingerichtet haben, wobei sie zusammen mit anderen Unternehmen eine “gemeinsame Meldestelle“ betreiben können, vgl. Art. 8 III WBRL/§ 12 HinSchG.\footnote{Rohrlich, Hinweisgeberschutzgesetz.}
Für die regelmäßige Kontrolle und Bearbeitung der eingehenden Meldungen muss dementsprechend eine zuständige Person oder Stelle benannt werden.\footnote{Rohrlich, Hinweisgeberschutzgesetz.}
Hierfür müssen nicht unerhebliche personelle Ressourcen aufgewendet werden.\\
Die Meldekanäle müssen zudem nach Art. 9 WBRL einige Anforderungen erfüllen. 
Unter anderem müssen diese so sicher konzipiert, eingerichtet und betrieben werden, dass die Vertraulichkeit der Identität des Hinweisgebers gewahrt bleibt, weiter muss dem Hinweisgeber der Eingang seiner Meldung innerhalb von sieben Tagen bestätigt werden und es muss eine Rückmeldung innerhalb von drei Monaten erfolgen.\footnote{Degenhart/Dziuba, BB 2021/9, 572.}
Sollte dies nicht gewährleistet sein, drohen den Unternehmen eine Geldbuße von bis zu 20.000 Euro, da eine Verletzung als Ordnungswidrigkeit gewertet wird.\footnote{Rohrlich, Hinweisgeberschutzgesetz.}\\
Ein weiterer Punkt, der arbeitsrechtlich zu Herausforderungen führen könnte, könnte die Beweislastumkehr aus Art. 21 V WBRL sein.
Dieses System kann potenziell von Arbeitnehmern ausgenutzt werden, indem diese sich durch einen rechtzeitigen „Hinweis“ zusätzlichen Kündigungsschutz verschaffen.\footnote{Dzida/Granetzny, NZA 2020, 1201.}
Der Arbeitgeber müsste nun erst einmal nachweisen, dass er den Arbeitnehmer nicht aufgrund des Hinweises kündigen wollte.
Dies ist vor allem bei wie der Probezeitkündigung, in denen eine rechtliche Notwendigkeit zur Begründung einer Maßnahme normalerweise gar nicht besteht, relevant.\footnote{Degenhart/Dziuba, BB 2021/9, 573.}
Der Arbeitgeber muss nun faktisch doch begründen, warum die Kündigung stattfand, um zu beiweisen, dass sie nicht auf dem Hinweis beruhte.\footnote{Degenhart/Dziuba, BB 2021/9, 573.}
Es ist zur Verhinderung eine umfassende Dokumentation der Unternehmen von zum Beispiel Mitarbeiterbewertungen, Karriereentwicklungen, erteilte Abmahnungen sowie bereits aufgetretene Probleme, notwendig, sodass gegebenenfalls im Kündigungsprozess nachgewiesen werden kann, dass dieser nicht mit dem vermeintlichen Whistleblowing zusammenhängt.\footnote{Dzida/Granetzny, NZA 2020, 1201.}
Mit dieser umfassenden Dokumentation ist aber eben auch ein enormer Aufwand verbunden, den die Arbeitgeber nun zusätzlich zu bewältigen haben.



\subsubsection{Datenrechtliche Herausforderungen}
Auch in Verbindung mit der Datenschutzgrundverordnung (DSGVO) kann es zu Spannungen kommen.
Gem. Art. 1 DSGVO zielt diese Verordnung darauf ab natürliche Personen bei der Verarbeitung personenbezogener Daten zu schützen.
Ferner sollen die Grundrechte und Grundfreiheiten natürlicher Personen und insbesondere deren Recht auf Schutz personenbezogener Daten geschützt werden.
Für die Verarbeitung personenbezogener Daten ist immer eine Rechtsgrundlage erforderlich.\footnote{Altenbach/Dierkes, CCZ 2020, 126.}
Bislang ergab sich diese in Bezug auf Hinweisgebersysteme gem. Art. 6 I 1 c) DSGVO nur für Unternehmen in ausgewählten Wirtschaftsbereichen.
Durch die Umsetzung der Whistleblowing-Richtlinie ergibt sich aus der nationalen Umsetzungsnorm zu Art. 8 WBRL eine entsprechende Rechtsgrundlage.\footnote{Altenbach/Dierkes, CCZ 2020, 126.}
In Deutschland ergibt sich diese aus § 12 HinSchG.
Hiervon können jedoch nur Unternehmen gebrauch machen, welche in den Anwendungsbereich der Whistleblowing-Richtlinie fallen.
Somit müssen vor allem Unternehmen mit weniger als 50 Mitarbeitern weiterhin gemäß Art. 6 I 1 f) DSGVO abwägen, ob die Verarbeitung zur Wahrung der berechtigten Interessen des Unternehmens erforderlich ist und die Interessen oder die Grundrechte der betreffenden Person nicht überwiegt.\footnote{Altenbach/Dierkes, CCZ 2020, 126.}\\
Ferner könnte es zu Konflikten mit Art. 14 DSGVO kommen.
In Art. 14 DSGVO wird die Art und der Umfang der Informationspflicht des Verantwortlichen gegenüber der betroffenen Person geregelt, wenn und soweit die personenbezogenen Daten nicht bei der betroffenen Person erhoben werden.\footnote{Paal/Pauly DSGVO, \textit{Paal/Hennemann}, Art. 14 Rn. 1.}
Die Ausübung dieser Informationspflicht hat gesetzlich innerhalb eines Monats nach Erlangung der personenbezogenen Daten zu erfolgen.\footnote{Paal/Pauly DSGVO, \textit{Paal/Hennemann}, Art. 14 Rn. 34.}\\
Eine solche Pflicht steht dem Sinn eines Hinweisgebersystems und den Interessen sowohl des Unternehmens als auch des Hinweisgebers entgegen.\footnote{Altenbach/Dierkes, CCZ 2020, 126.}
Einerseits ist es fraglich, wie die Informationspflicht mit dem Interesse des Hinweisgebers anonym zu bleiben vereinbar ist, und andererseits könnte eine vorzeitige Unterrichtung eine Warnfunktion haben und zu einer Verdunkelungsgefahr führen, weil es dem Beschuldigten zum Beispiel ermöglicht wird Beweismittel zu verändern oder auf Zeugen einzuwirken.\footnote{Altenbach/Dierkes, CCZ 2020, 126.}
Abhilfe wird hier durch Art. 14 V b) DSGVO geschaffen.
Dort wird geregelt, dass die Unterrichtung so lange hinausgezögert werden darf, wie das erhebliche Risiko besteht, dass infolge der fristgerechten Umsetzung die Untersuchung der gemeldeten Vorwürfe oder die Erhebung der erforderlichen Beweise gefährdet wird.\footnote{Altenbach/Dierkes, CCZ 2020, 126.}
Weiter kann die Anonymität des Hinweisgebers durch Art. 16 WBRL gesichert werden, welcher das Vertraulichkeitsgebot regelt. 
Der Schutz hieraus ist jedoch nicht absolut.\footnote{Dilling, CCZ 2019, 214.}
Zwar haben nach Art. 16 I WBRL die Mitgliedstaaten sicherzustellen, dass die Identität des Hinweisgebers und auch alle Informationen, aus denen sich die Identität ableiten ließe, ohne dessen ausdrücklicher Zustimmung keinen anderen Personen außer den befugten Mitarbeitern gegenüber offengelegt wird, jedoch darf sie nach Art. 16 II WBRL dann offengelegt werden, wenn dies nach Unionsrecht oder nationalem Recht eine notwendige und verhältnismäßige Pflicht im Rahmen von Untersuchungen durch Behörden darstellt.\footnote{Dilling, CCZ 2019, 214.}
Eine Offenlegung der Identität ist jedoch in der Regel vorerst nicht zu befürchten.\footnote{Altenbach/Dierkes, CCZ 2020, 126.}\\
Im Zusammenhang mit den Meldungen in Hinweisgebersystemen bildet  ferner die Löschungsverpflichtung aus Art. 17 I a) DSGVO eine Herausforderung.\footnote{Altenbach/Dierkes, CCZ 2020, 126.}
Nach Art. 17 I a) DSGVO müssen die personenbezogenen Daten gelöscht werden, wenn diese für die Zwecke, für die sie erhoben oder auf sonstige Weise verarbeitet worden sind, nicht mehr notwendig sind.\footnote{Paal/Pauly DSGVO, \textit{Paal}, Art. 17 Rn. 23.}\\
Ein mögliches Problem, welches sich daraus ergeben kann, wäre, dass zum Beispiel ein Unternehmen, dass pflichtgemäß einem Hinweis eines Mitarbeiters nachgeht, aber keine entsprechenden Verstöße feststellen konnte und schließlich der Löschungsverpflichtung nachkommt. 
Nun könnte dieser Mitarbeiter anschließend mit seinem Hinweis an die Öffentlichkeit gehen, wodurch dem Unternehmen medienwirksam unterstellt werden würde, dass es auf ernste Hinweise seiner Arbeitnehmer nicht reagieren würde.\footnote{Altenbach/Dierkes, CCZ 2020, 126.}
Eben dies würde zu einem erheblichen Imageschaden führen, und obwohl dem Unternehmen nichts vorzuwerfen ist, kann es sich nicht entlasten, da es durch die Löschung der Fallakte mit sämtlichen Dokumenten, Protokollen, etc. nichts vorweisen kann.\footnote{Altenbach/Dierkes, CCZ 2020, 126.}\\
Ein weiters Problem in Verbindung mit Art. 17 DSGVO ergibt sich aus dem allumfassenden Verbot von Repressalien aus Art. 19 WBRL und der Beweislastumkehr aus Art. 21 V WBRL.
Die Löschungsverpflichtung hat für Arbeitnehmer das Potenzial in Arbeitsgerichtsprozessen zu einem erfolgversprechenden Verteidigungsmittel zu werden.\footnote{Altenbach/Dierkes, CCZ 2020, 126.}
Kann der Arbeitnehmer beweisen, dass zuvor eine Meldung als Hinweisgeber getätigt zu haben, und behauptet er anschließend, nach der Löschung der Fallakte, dass die arbeitsgeberseitige Benachteiligung in Folge dieser Meldung geschehen ist, muss das Unternehmen nach Art. 21 V WBRL nachweisen, dass keine Repressalie i.S.d. Art. 19 WBRL vorliegt.\footnote{Altenbach/Dierkes, CCZ 2020, 126.}
Der Arbeitnehmer könnte sich mithin einen eigenen faktischen Kündigungsschutz herbeiführen.\footnote{Thüsing/Rombey, NZG 2018, 1001}
Kann das Unternehmen keinen Gegenbeweis vorlegen wird es zudem doppelt benachteiligt: Zum einen wird die benachteiligende Maßnahme gegen den Arbeitnehmer als unwirksam erklärt, und zum anderen ordnet Art. 23 I b), c) WBRL an, dass gegen Personen und Unternehmen, die Hinweisgebern Benachteiligungen auferlegen oder gegen diese mutwillig Gerichtsverfahren anstrengen, wirksame, angemessene und abschreckende Sanktionen verhängt werden sollen.\footnote{Altenbach/Dierkes, CCZ 2020, 126.}\\
Eine Lösung dieser Problematik ergibt sich zum Teil aus Art. 32 I a), ErwG 28, 29, 75, 78 DSGVO: die Fallakte kann zulässigerweise, statt ganz gelöscht, anonymisiert und pseudonymisiert werden.\footnote{Altenbach/Dierkes, CCZ 2020, 126.}
Damit diese dann dennoch Beweiskraft entfalten können, sollte der Sachverhalt, sowie die konkret ergriffenen Schritte so genau wie möglich dargestellt werden.\footnote{Altenbach/Dierkes, CCZ 2020, 126.}
Es sollte aber dennoch eine Abwägung stattfinden und nicht jede Akte vorbehaltlos gelagert werden: zu bedenken ist, dass sollten tatsächlich Rechtsverstöße des Unternehmens dort dokumentiert sein,  diese im Falle einer Beschlagnahmung durch Aufsichts- und Ermittlungsbehörden weiter verfolgt werden könnten.\footnote{Altenbach/Dierkes, CCZ 2020, 126.}
Es ist mithin zu empfehlen, dass die Akten nur so lange aufbewahrt werden, wie dies zur Wahrung der berechtigten Interessen des Unternehmens erforderlich ist.



\subsubsection{Strafrechtliche Herausforderungen}
Auch im Zusammenhang mit dem Strafrecht haben sich mit der neuen Richtlinie Herausforderungen ergeben.
Hinweisgeber könnten sich wegen der Offenlegung von Geschäftsgeheimnissen strafbar machen.
Diese Strafbarkeit ergabt sich früher aus § 17 I UWG.\footnote{Brockhaus, ZIS 3/2020, 103.}
Am 08.06.2016 wurde dann von der EU die Richtlinie zum Schutz von Geschäftsgeheimnissen (Richtlinie (EU) 2016/943) erlassen, welche dann am 18.04.2019 in Form des Gesetz zum Schutz von Geschäftsgeheimnissen (GeschGehG) in nationales Recht umgesetzt wurde.
Die Strafbarkeit von Hinweisgebern wird nun anhand des GeschGehG festgestellt.\\
Bei der Offenbarung von nicht offenkundigen Tatsachen, die sich auf rechtswidrige Praktiken eines Unternehmens (sog. „illegale Geheimnisse“) beziehen wird, bei der alten Rechtslage nach § 17 UWG, nach einer Ansicht vertreten, dass es mangels eines „berechtigten wirtschaftlichen Interesses“ bereits das Tatobjekt Geschäftsgeheimnis zu verneinen ist, sodass der Hinweisgeber es straflos weitergeben könnte (sog. Tatbestandslösung).\footnote{Reinbacher, KriPoZ 3/2019, 148.}
Nach der anderen Ansicht ist ein Geschäftsgeheimnis zwar anzunehmen, dann aber auf der Ebene der Rechtswidrigkeit eine Abwägung nach § 34 StGB vorzunehmen (Rechtswidrigkeitslösung).\footnote{Reinbacher, KriPoZ 3/2019, 148.}
Die §§ 17-19 UWG wurden nun aber aufgehoben und in § 23 GeschGehG überführt.\footnote{Brockhaus, ZIS 3/2020, 108.}
Eine Strafbarkeit für Whistleblower ergibt sich nun insbesondere aus § 23 I Nr. 3 i.V.m. § 4 II Nr. 3 GeschGehG, wonach sich derjenige strafbar macht, der zur Förderung des eigenen oder fremden Wettbewerbs, aus Eigennutz, zugunsten Dritten oder in der Absicht den Inhaber eines Unternehmens Schaden zuzufügen, als eine bei einem Unternehmen beschäftigte Person ein Geschäftsgeheimnis, das ihr im Rahmen des Beschäftigungsverhältnisses anvertraut worden oder zugänglich geworden ist, während der Geltungsdauer des Beschäftigungsverhältnisses offenlegt.\footnote{Reinbacher, KriPoZ 3/2019, 149.}
Fraglich ist nun, ob hierunter auch die illegalen Geheimnisse fallen sollen.
Hierfür muss zunächst geklärt werden, was unter einem Geschäftsgeheimnis zu verstehen ist.
Dieses ist in § 2 GeschGehG legal definiert.
Unter den Begriff des Geschäftsgeheimnisses fallen Informationen, die a) die weder insgesamt noch in der genauen Anordnung und Zusammensetzung ihrer Bestandteile den Personen in den Kreisen, die üblicherweise mit dieser Art von Informationen umgehen, allgemein bekannt oder ohne Weiteres zugänglich ist und daher von wirtschaftlichem Wert ist und b) die Gegenstand von den Umständen nach angemessenen Geheimhaltungsmaßnahmen durch ihren rechtmäßigen Inhaber ist und c) bei der ein berechtigtes Interesse an der Geheimhaltung besteht.\footnote{§ 2 Nr. 1 GeschGehG.}\\
Nach der ersten Tatbestandslösung scheitert eine Anwendung der §§ 2, 23 GeschGehG auf illegale Geheimnisse an dem Erfordernis des berechtigten Interesses an der Geheimhaltung.\footnote{Reinbacher, KriPoZ 3/2019, 150.}
Es ist aber zu beachten, dass Geschäftsdaten eines Unternehmens gegenüber Dritten nicht weniger schutzbedürftig sind, nur weil sich daraus z.B. eine Steuerhinterziehung ergibt; auch sollen vertrauliche Unterlagen über die Produktion eines Unternehmens wie z.B. Produktionspläne nicht der allgemeinen Kenntnis zustehen, nur weil sich daraus Rechtsverstöße ergeben.\footnote{Reinbacher, KriPoZ 3/2019, 152.}
Es würde sich hieraus für die betroffenen Unternehmen ein ungerechtfertigter Wettbewerbsnachteil ergeben.
Aus diesem Grund muss anerkannt werden, dass auch illegale Geheimnisse vom Geheimnisschutz umfasst.\footnote{Reinbacher, KriPoZ 3/2019, 152.}\\
Nach der zweiten Tatbestandslösung findet eine Rechtfertigung über § 5 Nr. 2 GeschGehG statt.\footnote{Reinbacher, KriPoZ 3/2019, 154.}
In § 5 Nr. 2 GeschGehG heißt es, dass die Offenlegung nicht unter die Verbote des § 4 GeschGehG fällt, wenn dies zum Schutz eines berechtigten Interesses erfolgt, insbesondere zur Aufdeckung einer rechtswidrigen Handlung oder eines beruflichen oder sonstigen Fehlverhaltens, wenn die Erlangung, Nutzung oder Offenlegung geeignet ist, das allgemeine öffentliche Interesse zu schützen.\footnote{§ 4 Nr. 2 GeschGehG.}
Somit sind bei Vorliegen des Tatbestandes sowohl zivilrechtliche Ansprüche aus §§ 6 ff GeschGehG, als auch erst recht die in § 23 GeschGehG vorgesehenen Straftaten ausgeschlossen.\footnote{Brockhaus, ZIS 3/2020, 111, 112.}
Aufgrund eines Übersetzungsfehlers war zunächst das Merkmal der Absicht erforderlich, dieses wurde aber wieder gestrichen.\footnote{Reinbacher, KriPoZ 3/2019, 155.}
Fraglich ist, inwieweit dennoch eine subjektive Komponente zu fordern ist.\footnote{Reinbacher, KriPoZ 3/2019, 156.}
Die Formulierungen „\textit{zum}“ und „\textit{zur}“ deuten auf einen subjektiven Einschlag hin\footnote{Reinbacher, KriPoZ 3/2019, 156.}, es handelt sich bei den Ausnahmetatbeständen um negative Tatbestandsmerkmale.\footnote{Brockhaus, ZIS 3/2020, 114.}
Soll der Hinweisgeber demnach bestraft werden, muss nachgewiesen werden, dass sich sein Vorsatz auf das Nichtvorliegen der Tatbestandsmerkmale erstreckt hat.\footnote{Brockhaus, ZIS 3/2020, 114.}
So muss demnach nachgewiesen werden, dass der Hinweisgeber zum Beispiel gerade nicht zur Aufdeckung einer Straftat gehandelt hat.
Es ist  jedoch nicht erforderlich, dass der Whistleblower in der Absicht handelt das allgemeine öffentliche Interesse zu schützen, es ist lediglich entscheidend, ob durch die Offenlegung das öffentliche Interesse geschützt werde.\footnote{Brockhaus, ZIS 3/2020, 114.} \\
Geht ein Hinweisgeber gutgläubig davon aus, zur Aufdeckung einer Straftat oder eines sonstigen Fehlverhaltens zu handeln, weil er entweder über Tatsachen irrt oder sie falsch bewertet, so ist sein Vorsatz eine Tat nach § 23 GeschGehG zu begehen ausgeschlossen.\footnote{Reinbacher, KriPoZ 3/2019, 157.}
Andererseits würde, wenn ein Ausnahmetatbestand zwar erfüllt ist, der Täter aber hiervon nichts wusste, eine Strafbarkeit wegen eines untauglichen Versuchs in Betracht kommen, vgl. § 23 V GeschGehG.\footnote{Brockhaus, ZIS 3/2020, 115.}
Im Endeffekt ist es mithin für den gutgläubigen Whistleblower nicht entscheidend, ob eine subjektive Komponente nun erforderlich ist oder nicht: denn entweder genügt das subjektive Handeln „zur Aufdeckung einer rechtswidrigen Handlung“ an sich, oder es mangelt am Tatbestandsvorsatz.\footnote{Reinbacher, KriPoZ 3/2019, 157.}\\
Im Rahmen des § 17 UWG hatte nach der herrschenden Meinung in Bezug auf das Whistleblowing eine Interessenabwägung und Verhältnismäßigkeitsprüfung des Einzelfalls stattzufinden.\footnote{Reinbacher, KriPoZ 3/2019, 157.}
Fraglich ist, ob dies mit dem neuen Gesetz immer noch gelten soll.
§ 5 Nr. 2 GeschGehG verlangt nun jedoch lediglich nach einer Geeignetheit, nicht aber nach einer Erforderlichkeit, eine Verhältnismäßigkeitsprüfung ist mithin nach dem Wortlaut nicht erforderlich.\footnote{Ohly, GRUR 2019, 441.}
Auch gesetzessystematisch liegt ein „berechtigtes Interesse“ insbesondere dann vor, wenn der Hinweisgeber „zur Aufdeckung einer rechtswidrigen Handlung“ tätig wird.\footnote{Reinbacher, KriPoZ 3/2019, 158.}
Ferner besteht zum einen die Möglichkeit, dass durch eine Verhältnismäßigkeitsprüfung die in § 5 GeschGehG vorgenommenen gesetzgeberischen Wertungen verwässert werden und so die Entscheidungsvorgänge für Betroffene kaum vorhersehbar werden und zum anderen würde die Komplexität der Normstruktur, die sowieso mit mehreren unbestimmten Rechtsbegriffen operiert, aufgebläht.\footnote{Brockhaus, ZIS 3/2020, 117.}
Eine Verhältnismäßigkeitsprüfung sollt mithin nur in Ausnahmefällen vorgenommen werden.\footnote{Brockhaus, ZIS 3/2020, 117.}\\
Handeln Hinweisgeber im guten Glauben daran, mit ihren Meldungen Missstände in Unternehmen aufzudecken, haben sie mithin nach dem GeschGehG keine strafrechtlichen Konsequenzen wegen des Verrats von Geschäftsgeheimnissen zu befürchten.