\subsection{Eignung des Whistleblowings zur Verhinderung von Straftaten innerhalb eines Unternehmens}

Zuletzt ist fraglich, ob durch das Whistleblowings überhaupt Straftaten innerhalb von Unternehmen verhindert werden können.
Durch Skandale wie beispielsweise der Finanzskandal Luxemburg-Leaks oder die Panama Papers wird deutlich, dass Hinweisgeber eine wichtige Rolle bei der Aufdeckung von Verstößen gegen das EU-Recht spielen können.\footnote{Europäisches Parlament, Whistleblower: Neue Vorschriften für EU-weiten Schutz von Informationen.}
Durch einen umfassenden Hinweisgeberschutz ließen sich in der EU jährlich im Bereich des öffentlichen Auftragswesens geschätzt 5,8 bis 9,6 Mrd. € an Ertragsausfällen einsparen.\footnote{Studie der Europäischen Kommission 2017, 15.}
Whistleblower können durch die Offenlegung von Informationen, die möglicherweise nicht ohne weiteres verfügbar sind rechtswidrige Verfahren aufdecken.
Hierfür sind jedoch klare und umfassende Besatimmungen zum Schutz von Hinweisgebern erforderlich, um Einzelpersonen zu ermutigen Betrug und anrere Arten von Fehlverhalten zu melden.\footnote{Studie der Europäischen Kommission 2017, 26.}
Denn Hinweisgeber schrecken aus Angst vor Repressalien häufig davor zurück Missstände zu melden.\footnote{Degenhart/Dziuba, BB 2021/9, 570.}
Ein Schutz der Hinweisgeber vor Repressalien förtert mithin die Aufklärung von Rechtsverstößen oder Missständen, bzw. vermeidet diese von Anfang an.\footnote{Degenhart/Dziuba, BB 2021/9, 571.}
Aus dem Grund, dass Arbeitnehmer eine geringere Hemmschwelle zum Melden von Missständen durch die neue Regelung haben, werden Unternehmen zunächst einmal eher auf diese aufmerksam. 
Zudem kommt die Verpflichtung aus Art. 9 WBRL, wonach sich die Unternehmen auch tatsächlich um die Meldung kümmern müssen. 
Der Hinweis muss mithin auf seinen Wahrheitsgehalt überprüft werden und, sollte er stimmen, der Rechtsverstoß beseitigt werden.
Das wird dazu führen, dass Unternehmen gezwungen sich sich rechtskonform zu verhalten, sollten sie verhindern wollen, dass ihr Verhatlen an die Öffentlichkeit gelangt, was zu erheblichen Einbußen und Repressalien führen könnte.\\
In Anbetracht dessen, könnten Unternehmen sogar eine höhere Hemmschwelle haben gegen Recht zu verstoßen, aus Angst, dass diese Missstände gegebenenfalls an die Öffentlichkeit gelangen.
Es kann also gesagt werden, dass durch den umfassenden Schutz von Hinweisgebern Straftaten innerhalb eines Unternehmens verhindert werden können, da Hinweisgeber nun eher offen dafür sind, etwas gegen diese Missstände zu sagen, wodurch Unternehmen gezwungen sind entprechende Maßnahmen einzuleiten.