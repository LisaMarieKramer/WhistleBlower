\subsection{Die EU Whistleblower Richtline}
\label{sec:Teilkapitel}

\subsubsection{Definition Whistleblowing}
Als Whistleblower wird eine Person bezeichnet, die Informationen über Missstände aus dem Innenbereich einer Organisation im öffentlichen Interesse einer Behörde bzw.  der breiteren Öffentlichekit (exterenes Whistleblowing) oder einer organisationsinternen Stelle (internes Whistleblowing) mitteilt.\footnote{Rechtswörterbuch, \textit{Kallos}, Whistleblower.} 
Meist handelt es sich bei einem Whistleblower um einen (etablierten oder ehemaligen) Mitarbeiter oder einen Kunden, welcher aus eigener Erfahrung berichtet.\footnote{Bendel, Whistleblowing.}
Das Whistleblowing ist von der bloßen Beschwerde über persönliche Umstände abzugrenzen, es setzt einen Umstand von allgemeinem Interesse voraus.\footnote{Bendel, Whistleblowing.}


\subsubsection{Aussagen der EU-Whistleblower Richtline}

Die Richtlinie (EU) 2019/1937 des Europäischen Parlaments und des Rates vom 23. Oktober 2019 zum Schutz von Personen, die Verstöße gegen das Unionsrecht melden, auch Whistleblowing-Richtlinie genannt, ist am 23.10.2019 erlassen worden.\footnote{Taschke/Pielow/Volk, NZWiSt 2021, 85.}
Sie wurde am 31.05.2023 durch das Hinweisgeberschutzgesetz (HinSchG) in deutsches Recht umgesetzt.
Durch die Richtlinie werden zwei Hauptziele verfolgt: zum einen sollen Whistleblower eine klar reglementierte Möglichkeit zur Meldung von Missständen haben und zum anderen sollen sie vor Repressialien geschützt werden.\footnote{Dzida/Granetzny, NZA 2020, 1201.}\\
Nun wird genauer auf die Einzelnen Aussagen der Richtlinie eingegangen.

\paragraph{Anwendungsbereich der Richtlinie}
Damit Whistleblower von dem Schutz durch die Richtline profitieren, müssen sie den sachlichen sowie den persönlichen Anwendungsbereich erfüllen.\footnote{Taschke/Pielow/Volk, NZWiSt 2021, 85.}\\
Der sachliche Anwendungsbereich wird in Art. 2 I der Whistleblower-Richtlinie (WBRL) geregelt. 
Der Anwendungsbereich begrenzt sich abschließend auf die dort aufgeführten Verstöße gegen Unionsrecht.\footnote{Taschke/Pielow/Volk, NZWiSt 2021, 85.}
Zum Beispiel sind dies Verstöße in den Bereichen öffentliche Auftragsvergabe, Finanzdienstleistungen, Geldwäsche und Terrorismusfinanzierung, Produktsicherheit, Verkehrssicherheit, Umweltschutz, kerntechnische Sicherheit, Lebensmittel- und Futtermittelsicherheit, Tiergesundheit und Tierschutz, öffentliche Gesundheit, Verbraucherschutz, Schutz der Privatsphäre, Datenschutz, Sicherheit von Netz- und Informationssystemen, EU-Wettbewerbsvorschriften, Körperschaftssteuervorschriften sowie Verstöße gegen die finanziellen Interessen der EU.\footnote{Dzida/Granetzny, NZA 2020, 1201.}
So fällt bloßes unethisches aber nicht rechtswidriges Fehlverhalten nicht in den Anwendungsbereich der Richtlinie.\footnote{Taschke/Pielow/Volk, NZWiSt 2021, 85.}
Ein größerer Anwendungsbereich war aufgrund der beschränkten Gesetzgebungskompetenz der EU nicht möglich, nach Art. 2 II WBLR bleibt es den Mitgliedstaaten jedoch freigestellt den Schutz in Bezug auf Bereiche oder Rechtsakte auszudehnen, die nicht unter Absatz 1 fallen.\footnote{Dzida/Granetzny, NZA 2020, 1201.}
Der deutsche Gesetzgeber ergänzte die Richlinie durch § 2 I Nr. 1, 2 und 10 HinSchG zum einen um „Verstöße, die strafbewehrt sind“, und zum anderen um „Verstöße, die bußgeldbewehrt sind, soweit die verletzte Vorschrift dem Schutz von Leben, Leib oder Gesundheit oder dem Schutz der Rechte von Beschäftigten oder ihrer Vertretungsorgane dient“.\footnote{Dzida/Seibt, NZA 2023, 657.}

\bigbreak

Der persönliche Anwendungsbereich wird in Art. 4 WBRL geregelt.
Aus den Wortlaut ergibt sich, dass der Hinweisgeber „im privaten oder im öffentlichen Sektor tätig“ sein und „im beruflichen Kontext Informationen über Verstöße erlangt haben“ muss.
Es sind demnach im Grunde alle Personen geschützt, die im weitesten Sinne in einer “arbeitsbezogenen Verbinung“ mit dem Unternehmen stehen.\footnote{Dzida/Granetzny, NZA 2020, 1201.}
Das Motiv des Hinweisgebers spielt keine Rolle.\footnote{Schmolke, NZG 2020, 5.}


\paragraph{Voraussetzungen für den Schutz}
In Art. 6 der Whistleblower-Richtlinie wird geregelt, welche Voraussetzungen für den Schutz erfüllt sein müssen.
Hiernach hat ein Hinweisgeber für einen Anspruch auf Schutz, sofern er im Zeitpunkt der Meldung im Hinblick auf die Umstände und die verfügbaren Informationen mit hinreichendem Grund davon ausgehen durfte, dass die gemeldete Information zutrifft und einen Verstoß darstellt, der in den Anwendungsbereich der Richtlinie fällt, vgl. Art. 6 I a)WBRL.\footnote{Taschke/Pielow/Volk, NZWiSt 2021, 85.}
Der Hinweisgeber muss demanch gutgläubig sein, wer also bewusst falsche oder irreführende Informationen meldet ist nicht schutzbedürftig.\footnote{Degenhart/Dziuba, BB 2021/9, 573.}\\
Die Meldung kann zunächst intern, also an eine geeignete Stelle inerhalb des Unternehmens, oder extern an die zuständige Behörde geschehen, vgl. Art 6 I b) WBRL i.V.m. Art. 7, Art. 10 WBRL.\footnote{Schmolke, NZG 2020, 5.}
Die interne und die externe Meldung stehen gleichrangig nebeneinander.\footnote{Dzida/Granetzny, NZA 2020, 1201.}
Höhere Anforderung an den Schutz durch die Richtlinie werden dann gestellt, wenn der Hinweisgeber sich direkt an die Öffentlichekit wendet, vgl. Art. 15 WBRL.\footnote{Schmolke, NZG 2020, 5.}
So muss der Hinweisgeber entweder den Verstoß zunächst intern oder extern gemeldet haben, ohne dass innerhalb der in Art. 11 WBRL festgelegten Höchstfristen von drei bzw. sechs Monaten geeignete Maßnahmen ergriffen worden sind.
Andere Situationen in denen der Hinweisgeber sich direkt an die Öffentlichekit wenden kann sind wenn er davon ausgehen durfte, dass der Verstoß eine „unmittelbare oder offenkundige Gefährdung des öffentlichen Interesses“ darstellt, oder aufgrund der besonderen Umstände des Falls geringe Aussichten bestehen, dass wirksam gegen den Verstoß vorgegangen wird, etwa bei Kollusion zwischen dem Delinquenten und der Behörde (Art. 6 I Buchst. b iVm Art. 15 I WBRL).\footnote{Schmolke, NZG 2020, 5.}
Die Meldung an die Öffentlichekit ist mithin als \textit{ultima ratio} subsidiär\footnote{Dzida/Granetzny, NZA 2020, 1201.} und der Hinweisgeber sollte sich zunächst an die eingerichteten internen oder externen Stellen richten.

\paragraph{Schutzmaßnahmen}
Sind die sachlichen und persönlichen Voraussetzungen erfüllt, wird der Schutz der Hinweisgeber vor allem durch das Verbot jeglicher Repressalien aus Art. 19 ff WBRL gewährleistet.\footnote{Dzida/Granetzny, NZA 2020, 1201.}
Eingeschlossen hierin ist ein Verbot für die Mitgliedstaaten Repressalien nur anzudrohen oder zu versuchen, vgl. Art. 19 WBRL.
Art. 19 WBRL stellt zur besseren Einschätzung einen Katalog an Regelbeispielen für mögliche Repressalien zur Verfügung.
Beispiele hieraus sind unter anderem Suspendierung oder Kündigung, Herabstufung oder Versagung einer Beförderung, Aufgabenverlagerung, negative Leistungsbeurteilung, sowie Nötigung und Diskriminierung.\\
Dieses Verbot wird durch die in Art. 21 V WBRL geregelte Beweislastumkehr verstärkt.
Hiernach muss nun das Unternehmen beweisen, dass es sich bei einer arbeitsrechtlichen Maßnahme nicht um eine unzulässige Repressialie i.S.d. Art. 19 WBRL handelt.\footnote{Dzida/Granetzny, NZA 2020, 1201.}
Somit ist nicht mehr der Arbeitnehmer in der Verantwortung darzulegen, dass eine arbeitsrechtliche Maßnahme aufgrund des Whistleblowings erfolgt ist.
Dies soll potentiellen Beweisschwierigkeiten, die sich für den Hinweisgeber ergeben könnten, entgegen wirken.\footnote{Taschke/Pielow/Volk, NZWiSt 2021, 85.}
Das Verbot jeglicher Repressalien wird ferner durch Art. 23 WBRL untermauert, wonach die Mitgliedstaaten “wirksame, angemessene und abschreckende“ Sanktionen für juristische sowie natürliche Personen festlegen müssen, die Meldungen behindern, Repressalien ergreifen, mutwillig Gerichtsverfahren gegen Hinweisgeber oder geschützte Dritte anstregen oder gegen das Vertraulichkeitsgebot im Rahmen des Meldeverfahrens verstoßen.\footnote{Schmolke, NZG 2020, 5.}
Somit wird sichergestellt, dass Hinweisgeber nach ihrer Meldung keine Konsequenzen zu befürchten haben.

\paragraph{Meldewege}
Potentiellen Hinweisgeber sollen nicht bloß geschützt werden, ihnen soll auch eine effektive Meldeinfrastruktur zur Verfügung gestellt werden.\footnote{Schmolke, NZG 2020, 5.}
Es wird zwischen internen und externen Meldekanälen unterschieden.\footnote{Schmolke, NZG 2020, 5.}
Art. 8 III WBRL gibt vor, dass Unternehmen mit 50 oder mehr Arbeitnehmern einen internen Meldekanal nach Art. 8 I errichten müssen.\footnote{Taschke/Pielow/Volk, NZWiSt 2021, 85.}
Diese müssen nach Art. 9 I a) WBRL zur Gewährleistung der Vertraulichkeit der Identität des Hinweisgebers und erwähnter Personen entsprechend sicher konzipirt, implementiert und betrieben werden.\footnote{Taschke/Pielow/Volk, NZWiSt 2021, 85.}
Zudem muss das Unternehmen dem Hinweisgeber nach Art. 9 I b) WBRL innerhalb von sieben Tagen den Eingang der Meldung bestätigen und ihm nach spätestens drei Monaten berichten, wie mit seinem Hinweis umgegangen wurde und was für Maßnahmen ergriffen worden sind.\footnote{Dzida/Granetzny, NZA 2020, 1201.}
Es wird somit sichergesetellt, dass den Meldungen der Hinweisgeber auch nachgegangen wird.\\
Für die externen Meldekanäle müssen gemäß Art. 10 WBRL durch die Mitgliedstaaten zuständige Behörden benannt werden, die befugt sind Meldungen entgegenzunehmen, Rückmeldung zu geben und entsprechende Folgemaßnahmen zu ergreifen.\footnote{Dzida/Granetzny, NZA 2020, 1201.}
Zu diesem Zweck müssen die Behörden mit angemessenen Ressourcen ausgestattet werden, vgl. Art. 11 I WBRL.
Zur Wahrung des Vertraulichkeitsgebots müssen die Meldekanäle grundsätzlich so konzipirt sein, dass Unbefugte keinen Zugriff auf die Daten erhalten können, vgl. Art 16 WBRL.\footnote{Dzida/Granetzny, NZA 2020, 1201.}



\subsubsection{Beeinflussung der Verabschiedung durch Entwicklungen in den USA}

Das Whistleblowing-Recht in den USA hat dort eine langjährige Tradition.\footnote{Gerdemann, NZA-Beilage 2020, 43.}
Hintergrund ist das Bedürfnis des dortigen Bundesgesetzgebers, trotz der vergleichsweise beschränkten eigenen Administrativressourcen, Verstöße gegen bundesrechtliche Normen in Erfahrung zu bringen und hierdurch eine wirksame praktische Durchsetzung seiner seit dem letzten Jahrhundert sukzessive expandierenden Legislativkompetenzen zu erwirken.\footnote{Gerdemann, NZA-Beilage 2020, 43.}
Dieser Regulierungsmechanismus der USA führte zunehmend zu Erfolgen, sodass sich die Europäische Union das Prinzip des Whistleblowings zu eigen gemacht hat.\footnote{Gerdemann, NZA-Beilage 2020, 43.}
Die USA fungierte mithin als legislatives Vorbild.
Vor allem zwei Akte zogen die Aufmerksamkeit der Europäischen Unions auf sich: Der Sarbanes-Oxley Act von 2002 (SOX) und der Dodd-Frank Act von 2010.\footnote{Gerdemann, SR 1, 2021}\\
Der erst genannte ist ein vom amerikanischen Kongress verabschiedetes Bundesgesetz mit dem Ziel, das Vertrauen der Anleger in die Richtigkeit und Verlässlichkeit der veröffentlichten Finanzdaten von Unternehmen wiederherzuestellen.\footnote{CC, \textit{Obermayr}, § 44 Rn. 111.}
Er wurde als Reaktion auf die Bilanzskandale erlassen.\footnote{Schürrle/Fleck,CCZ 2011, 218.}
Unter anderem wird den betroffenen Unternehmen die Pflicht zur Errichtung eines internen Kontrolsystems sowie die Etablierung eines Beschwerdemanagements (Whistleblower-System) auferlegt.\footnote{Handbuch VertriebsR, \textit{Passarge}, § 79 Rn. 120.}
Von dem Act betroffen sind alle Unternehmen, deren Aktien an der New York Stock Exchange gehandelt werden; hierunter sind auch 23 deutsche Unternehmen.\footnote{CC, \textit{Obermayr}, § 44 Rn. 111.}
Mithin ist der SOX grundsätzlich unabhängig vom Sitz des Unternehmens anwendbar\footnote{Handbuch VertriebsR, \textit{Passarge}, § 79 Rn. 119.}, hierdurch ergibt sich bereits eine Relevanz für die EU.
Bereits im SOX wurden Hinweisgeber vor Diskriminierung in Form von Entlassung, Zurückstufung oder ähnlichem geschützt, bei Verstoß konnten den Arbeitnehmern Schadenersatzansprüche zustehen.\footnote{Schürrle/Fleck,CCZ 2011, 218.}\\
Der Dodd-Frank Act wurde schließlich als Reaktion auf den Finanzskandal 2008 erlassen.\footnote{Schürrle/Fleck,CCZ 2011, 218.}
Durch diesen Act sollte das Antidiskriminierungsrecht des SOX noch verbessert werden und zudem wurde mit dem „SEC Office of the Whistleblower“ eine externe Meldebehörde etabliert.\footnote{Gerdemann, SR 1, 2021}
Durch den Dodd-Frank Act wurde also der Schutz der Hinweisgeber vor Vergeltung ausgeweitet, hierfür wurden Durchführungsbestimmungen vorgegeben, zudem wurde der Anwendungsbereich auf nicht-börsengelistete Tochterunternehmen ausgeweitet.\footnote{Schürrle/Fleck,CCZ 2011, 218.}
Die Europäische Union sah sich mit ähnlichne Herausforderungen wie das Federal Government der USA konfrontiert.\footnote{Gerdemann, SR 1, 2021}
So ist es nicht verwunderlich, dass die Whistleblowing-Richtlinie der Europäischen Union den Regelungen aus dem USA zum Teil ähnelt.\\
Zudem kommt der Faktor, dass durch internationale Whistleblowing-Fälle das öffentliche Interesse am Phänomen des Whistleblowings gestiegen ist.\footnote{Gerdemann, SR 1, 2021} 
Beispiele für bekannte Fälle wären unter anderem Edward Snowden, Luxemburg-Leaks oder die Panama Papers.
Durch diese wurde verstärkt deutlich, dass Hinweisgeber eine wichtige Rolle bei der Aufedeckung von Verstößen gegen das EU-Recht spielen können, die das öffentliche Interesse und das Wohlergehen der Bürger und der Gesellschaft schädigen.\footnote{Yakimova, Whistleblower: Neue Vorschriften für EU-weiten Schutz von Informationen.}
Trotz der großen unionsrechtlichen Relevanz hatten vor Erlass der Richtlinie nur 10 EU-Länder einen umfassenden Schutz für Hinweisgeber.\footnote{Yakimova, Whistleblower: Neue Vorschriften für EU-weiten Schutz von Informationen.}
Eine breit angelegte, sektorübergreifende Regulierung des Whistleblowings in Europa schien dennoch zunächst eher unwahrscheinlich, bis sich eine Koalition aus staatlichem Rechtsdurchsetzungsinteresse und zivilgesellschaftlichem Engagement bildete.\footnote{Gerdemann, SR 1, 2021} 
