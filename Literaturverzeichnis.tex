\section{Literaturverzeichnis}

\begin{longtable}{R{5 cm}R{8,5 cm}}

    \textbf{Altenbach, Thomas/ Dierkes, Kevin} & Corporate Compliance Zeitschrift, 2020, S. 126 ff, „EU-Whistleblowing-Richtlinie und DSGVO“ (zit.:Altenbach/Dierkes, CCZ 2020, 126.) \\
    \\

    \textbf{Bayreuther, Frank} & Neue Zeitschrift für Arbeitsrecht - Beilage, 2022, S. 20 ff,  „Whistleblowing und das neue Hinweisgeberschutzgesetz“ (zit.: Bayreuther, NZA-Beilage 2022, 20.) \\
    \\

    \textbf{Bendel, Oliver} & Gabler Wirtschaftslexikon, „Whistleblowing“, Fundort: https://wirtschaftslexikon.gabler.de/definition/
    
    whistleblowing-53526, Abrufdatum: 22.05.2024  (zit.: Bendel, Whistleblowing.) \\
    \\

    \textbf{Brockhaus, Robert} & Zeitschrift für Internationale Strafrechtswissenschaft, 2020, Ausgabe 3, S. 102 ff, „Das Geschäftsgeheimnisgesetz - Zur Frage der Strafbarkeit von Hinweisgebern unter Berücksichtigung der Whistleblowing-Richtlinie“ (zit.: Brockhaus, ZIS 3/2020, S.) \\
    \\

    \textbf{Dilling, Johannes} &  Corporate Compliance Zeitschrift, 2019, S. 214 ff, „Der Schutz von Hinweisgebern und betroffenen Personen nach der EU-Whistleblower-Richtlinie“ (zit.: Dilling, CCZ 2019, 214.) \\
    \\

    \textbf{Degenhart, Maximilian/ Dziuba, Anne} & Betriebs Berater, 9. Auflage, 01.03.2012: „Die EU-Whistleblower-Richtlinie und ihr earbeitsrechtlichenA uswirkungen“ (zit.: Degenhart/Dziuba, BB 2021/9, S.) \\
    \\

    \textbf{Dzida, Boris/ Granetzny, Thomas} & Neue Zeitschrift für Arbeitsrecht, 2020, S. 1201 ff, „Die neue EU-Whistleblowing-Richtline und ihre Auswirkungen auf Unternehmen“ (zit.: Dzida/Granetzny, NZA 2020, 1201.) \\
    \\ 

    \textbf{Dzida, Boris/ Seibt, Christopj} & Neue Zeitschrift für Arbeitsrecht, 2023, S. 657 ff, „Neues Hinweisgeberschutzgesetz: Analyse und Antworten auf Praxisfragen“ (zit.: Dzida/Seibt, NZA 2023, 657.) \\
    \\

    \textbf{Europäische Kommission} & Estimating the economic benefits of whistleblower protection in public procurement : final report, Publications Office, 2017, https://data.europa.eu/doi/10.2873/125033 (zit.: Studie der Europäischen Kommission 2017, S.) \\
    \\
    
    \textbf{Gerdemann, Simon} & Neue Zeitschrift für Arbeitsrecht Beilage, 2020, S. 43 ff, „Whistleblower als Agenten des Europarechts“ (zit.: Gerdemann,NZA-Beilage 2020, 43.) \\
    \\

    \textbf{Gerdemann, Simon} & Soziales Recht, Vol. 11, No. 1, Februar 2021 „Whistleblowing, quo vadis? Die Europäische Whistleblower-Richtlinie und ihre Umsetzung in deutsches Recht – Teil 1: Die Richtlinie und ihre neuralgischen Punkte“ (zit.: Gerdemann, SR 1, 2021) \\
    \\

    \textbf{Martinek, Micheal/ Semler, Franz-Jörg/ Flohr, Eckhard} & Handbuch des Vertriebsrechts, 4. Auflage, München 2016 (zit.: Handbuch VertriebsR, \textit{Bearbeiter}, § Rn.) \\
    \\

    \textbf{Meyer-Krumenacker, Astrid} & QZ-online.de „Risiken des Hinweisgeberschutzgesetzes“, 02.11.2023 Fundort: https://www.qz-online.de/a/fachartikel/risiken-des-hinweisgeberschutzgesetzes-5186581, Abrufdatum: 28.05.2024 (zit.: Meyer-Krumenacker, Risiken des Hinweisgeberschutzgesetzes.) \\
    \\

    \textbf{Ohly, Ansgar} & Gewerblicher Rechtsschutz und Urheberrecht, Ausgabe 5, 2019: „Das neue Geschäftsgeheimnisgesetz im Überblick“ (zit.: Ohly, GRUR 2019, 441.) \\
    \\

    \textbf{Paal, Boris P./ Pauly, Daniel A.} & Beck'sche Kompakt-Kommentare Datenschutz-Grundverordnung, 3. Auflage, München 2021 (zit.: Paal/Pauly DSGVO,\textit{Bearbeiter}, Art. Rn.) \\
    \\

    \textbf{Reinbacher, Tobias} & Kriminalpolitische Zeitschrift, Ausgabe 3, 2019: „Der neue Straftatbestand des § 23 GeschGehG und das Whistleblowing“ (zit.: Reinbacher, KriPoZ 3/2019, S.) \\
    \\

    \textbf{Rohrlich, Michael} & lexware „Hinweisgeberschutzgesetz – was ist das und welche Maßnahmen müssen Unternehmen umsetzen?“, 16.05.2024, Fundort: https://www.lexware.de/wissen/unternehmensfuehrung/
    
    hinweisgeberschutzgesetz-das-muessen-unternehmen-jetzt-tun/, Abrufdatum: 28.05.2024 (zit.: Rohrlich, Hinweisgeberschutzgesetz.) \\
    \\

    \textbf{Schürrle, Thomas/ Fleck, Franziska} & TCorporate Compliance Zeitschrift, 2011, S. 218 ff, „„Whistleblowing Unlimited” – Der U.S. Dodd-Frank Act und die neuen Regeln der SEC zum Whistleblowing” (zit.: Schürrle/Fleck, CCZ 2011, 218.) \\
    \\

    \textbf{Schmolke, Klaus Ulrich} & Neue Zeitschrift für Gesellschaftsrecht, 2020, S. 5 ff, „Die neue Whistleblower- Richtlinie ist da! Und nun?“ (zit.: Schmolke, NZG 2020, 5) \\
    \\

    \textbf{Taschke, Jürgen/ Pielow, Tobias/ Volk, Ela} & Neue Zeitschrift für Wirtschafts-, Steuer- und Unternehmensstrafrecht, 2021, S. 85 ff, „Die EU-Whistleblower-Richtlienie - Herausfprderungen für die Unternehmenspraxis“ (zit.: Taschke/Pielow/Volk, NZWiSt 2021, 85.) \\
    \\

    \textbf{Thüsing, Gregor/ Rombey, Sebastian} & Neue Zeitschrift für Gesellschaftsrecht, 2018, S. 1001 ff, „Nachdenken über den Richtlinienvorschlag der EU-Kommission zum Schutz von Whistleblowern“ (zit.: Thüsing/Rombey, NZG 2018, 1001.) \\
    \\

    \textbf{Yakimova, Yasmina} & Whistleblower: Neue Vorschriften für EU-weiten Schutz von Informationen, Pressemitteilung Europäisches Parlament vom 16.04.2029; Fundort: https://www.europarl.europa.eu/news/de/press-room/20190410IPR37529/whistleblower-neue-vorschriften-fur-eu-weiten-schutz-von-informanten; Abrufdatum: 27.05.2024 (zit.: Yakimova, Whistleblower: Neue Vorschriften für EU-weiten Schutz von Informationen.) \\
    \\

    \textbf{Weber, Klaus} & Rechtswörterbuch, 32. Edition, München 2024 (zit.: Rechtswörterbuch, \textit{Bearbeiter}, Eintrag.) \\
    \\
    
\end{longtable}
